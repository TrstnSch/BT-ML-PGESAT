\chapter{Experiments and Results}
\label{ch:Experiments}

TODO: In this chapter we introduce all performed experiments. This includes a general experimetal setup, as well as more detailed experiment settings including their results.

Since our work is twofold, we start with the experiments regarding the replication of the PGExplainer with a changed downstream model architecture and a focus on the inductive setting in Section \ref{}. Next, we describe the experiments performed on the adapted explainer framework to generate explanations for the NeuroSAT model predictions of unsatisfiable problems in Section \ref{}.

\section{Replication of PGExplainer}
All experiments regarding the replication of the presented explainer model. This includes a common experimental setup, including the datasets, corresponding downstream models, as well as hyperparameter searches. In Section xy we replicate the experiments in the inductive setting from the original paper. We also include a study of the effect of using a larger set of training data, as the original proposes that only few training instances are necessary for the explainer to generalize well. Furthermore, we perform the original quantitative experiment in the collective setting with the same hyperparameters for better comparability, as this was the core study of the original paper. In Section xy we run an experiment specific to the BA-2Motif dataset, as we found that it behaves opposite to the expectations. Later, we study the effects of the specific node sets that were used in the original codebase in Section xy. Lastly, we qualitatively evaluate the explanations provided by our reimplemented model (see Section xy).

\subsection{Common experimental setup}
In this section we describe the common setup for the experiments that we perform on PGExplainer.

\textbf{Datasets}
For our reimplementation we perform the experiments on the same datasets used in the original. These were constructed by the authors similarly to the ones used in the baseline GNNExpplainer. Four synthetic datasets were used for the node classification tasks. For the graph classification task the authors provide one synthetic dataset as well as the real-world dataset MUTAG. The synthetic datasets are constructed by creating a base graph and attaching motifs to random nodes of the base graph. These motifs determine the labels of the nodes or graphs, depending on the task at hand, and therefore serve as the ground truth explanations that the explainer shall detect. Statistics of each dataset can be found in table \ref{tab:dataset-statistics}. We will give a short description of each dataset.

Since three of the synthetic datasets use a Barabási-Albert (BA) graph as a base, we briefly introduce the BA model. The BA model generates scale-free networks that grow over time. Starting with an initialization network of $m_0 \geq m$ nodes, at each step a new node is added and connected to $m$ of the nodes already existing in the graph. The probability for each node to be selected as a neighbor depends on its degree, leading to a higher probability for nodes that already have a high degree rather than nodes with a low degree \cite{albert2002statistical}.

BA-Shapes is the first node dataset that consists of a single BA-graph with 300 nodes and 80 "house" motifs - five nodes resembling the shape of a house (TODO: see xy). Base graph nodes are labeled with 0 while nodes at the top/middle/bottom of the "house" are labeled with 1,2,3, respectively. The top node of each house motif is attached to a random base graph node. Additional edges are added for perturbation. Each node is assigned a 10-dimensional feature vector of 1s.
BA-Community consists of two unified BA-Shapes graphs (TODO: connected how). The features of the nodes are sampled from two Gaussian distributions. Nodes are labeled as in BA-Shapes for each community respectively, leading to 8 classes in total.
Tree-Cycles uses an 8-level balanced binary tree as a base graph. 80 cycle motifs, consisting of a 6 node cycle, are attached to random nodes from the base graph. Node features are assigned as a 10-dimensional vector of 1s. A node of the base graph is labeled as 0 and a motif node is labeled as 1.
The Tree-Grid dataset is assembled in the same way as Tree-Cycles, with the difference that the motifs are 3-by-3 grids. Node features and labels also follow the same procedure.
BA-2Motif is the first graph dataset with 800 graphs. Each of these graphs is obtained by attaching either a "house" or a cycle as a motif to a base BA graph with 20 nodes. According to the attached motif the graphs are assigned one of two labels, with 0 and 1 implying a house and circle, respectively (TODO: CHECK).
The real-world dataset MUTAG contains $4,337$ molecule graphs that are assigned to one of 2 classes, depending on the molecules mutagenic effect \ref{}. Following \ref{}, carbon rings with chemical groups $NH_2$ or $NO_2$ are known to be mutagenic, with carbon rings in general existing in both mutagenic and non-mutagenic graphs. The authors thus propose treating the carbon ring as a shared base graph and $NH_2$ and $NO_2$ as motifs for mutagenic graphs. Since there are no explicit motifs for the non-mutagenic graphs, these grapgs are not considered in PGExplainer.

NOTE that in the collective setting used in the original paper the explainer is trained and evaluated on the same data. This data is further reduced by only using graphs and nodes that contain a ground truth motif. This makes sense for evaluation, since the AUROC cannot be calculated for ground truths with only one class present. However, the authors do not specify why the training is performed only on these instances. Therefore, only the mutagenic graphs where either $NH_2$ or $NO_2$ are present are selected for the MUTAG experiment. In the node classification experiments the node sets used for training and evaluation were further finetuned per dataset. This leads to a selection of either all nodes that are part of a motif, or only one node per motif. This is also left unexplained by the authors.



\textbf{Downstream models used}: train/test/validation split. TRAINING SET SIZE = 30 FOR COMPARABILITY WITH ORIGINAL! The node sets of motif nodes used for training and evaluating the explainer are extracted from the codebase. The same is done for the mutagenic graphs with existing ground truth, namely TODO. Similar accuracies achieved of at least $85\%$ (see table \ref{tab:compact-accuracy})

\textbf{Training procedures}: How long you trained, on which hardware (especially if relevant), early stopping criteria, etc.

\textbf{Evaluation metrics}: AUROC score evaluation for quantitative evaluation; Paper does not describe calculation, code varies between datasets. Since explanations are collective, calculating global AUROC is valid, because all edges share the same network parameters (?). We propose consistently calculating local AUROC for each instance and meaning over the dataset, as we use inductive setting. Additionally, AUROC for (sub-)graphs that contain only or no motif edges cannot be calculated individually and is therefore excluded/skipped. This is not regarded in the global calculation. This is not discussed in the original paper.

Qualitative evaluation using the top-$k$ explanation mask edges with highest importance scores. Compared to ground truht that contains $k$ edges. $k$ can be understood as a parameter. For node classification tasks, we only show the computational graph of the node to be explained, as this is known to be relevant for the prediction \cite{}, where the top$-k$ edges are highlighted as explanation.

Efficiency evaluation performed by authors, big increase in eff. since training time is disregarded when compared to GNNExplainer. However, exact method for this not clear. We follow the calculation of the RE-PGE to compare the average time needed for a trained explainer model to return an explanation for any one instance.

\textbf{Hyperparameter tuning}: Hyperparameter tuning using grid search for each dataset

\textbf{Baselines}: Original in both inductive and collective setting, RE-PGE in collective setting

\textbf{Experimental protocol}: Averaged over 10 seeds account for randomness, similar to original.


We run additional experiments to evaluate the effects of certain inconsistencies. Therefore, we change the motif nodes that are used for training/evaluating to include all or one select motif node, depending on the dataset. We use the same hyperparameters used for the previous experiments. Besides this change, we follow the experimental setup to evaluate whether a reasoning behind the node selection can be made out.


\begin{table}[h]
    \centering
    \scriptsize
    \begin{tabular}{l|cccc|cc}
    \hline
    \textbf{} & \textbf{BA-Shapes} & \textbf{BA-Community} & \textbf{Tree-Cycles} & \textbf{Tree-Grid} & \textbf{BA-2motifs} & \textbf{MUTAG} \\
    \hline
    \#graphs & 1 & 1 & 1 & 1 & 1,000 & 4,337 \\
    \#nodes  & 700 & 1,400 & 871 & 1,231 & 25,000 & 131,488 \\
    \#edges  & 4,110 & 8,920 & 1,950 & 3,410 & 51,392 & 266,894 \\
    \#labels & 4 & 8 & 2 & 2 & 2 & 2 \\
    \hline
    \end{tabular}
    \caption[Statistics of PGExplainer datasets]{Dataset statistics for Node and Graph Classification tasks.}
    \label{tab:dataset-statistics}
    \end{table}

\bigskip

\subsection{PGExplainer in the inductive setting}

\textbf{Experimental setup}


TODO: WE USE PGEXPLAINER IN INDUCTIVE SETTING, OPPOSED TO COLLECTIVE SETTING USED IN ORIGINAL/GNNExplainer. INDUCTIVE SETTING EXPERIMENTS ALSO PERFORMED IN PGEXPLAINER!

$N=30$ training instances for all datasets.

TODO: LOW INFERENCE TIMES BECAUSE OF CPU INSTEAD OF GPU!?
Experiments ran on AMD Ryzen 7 2700X


NOTE: Tree-Grid with original experimental setup (all motif nodes, 30 training instances) leads to mean of 0.5 for almost all hyperparam settings tried. Size reg of 0.05 (as in original) leads to all edges being assigned a values of one!

We conduct an experiment on the performance with only select motif nodes! only taking 30 random nodes from all motif nodes may lead to a selection of mostly nodes where not even the complete motif, and even less probable edges outside the motif, are contained.
%"In this section, we empirically demonstrate the effectiveness of PGExplainer in the inductive setting. In the inductive setting, we select α instances for training, (N − α)/2 for validation, and the rest for testing. α is ranged from [1,2,3,4,5,30]. Note that, with α =1, our method degenerates to the single-instance explanation method. Recall that to explain a set of instances, GNNExplainer first detects a reference node and then computes the explanation for the reference node. The explanation is then generalized to other nodes with graph alignment(SOURCE GNNExplainer)"

\textbf{Hyperparameter search}

As found by Holdijk et al.\ref{} the PGExplainer is very sensitive to hyperparameter settings on each dataset. Therefore, we conduct hyperparameter searches for each of the datasets to obtain best performing explainers. We follow Liashchynskyi et al. \cite{liashchynskyi2019grid} to perform grid searches over the parameter space that we define as an extended combination of the setting used in the original \cite{}, as well as the configs provided in Replication study \cite{}. More details can be found in Appendix\ref{}. For our hyperparameters $\theta_1,\theta_2,...,\theta_n$ we define corresponding sets $S_1,S_2,...,S_n$ of possible values. The grid search finds the best model with respect to the mean of the local ROC-AUC scores of all validation instances over all combinations $(\theta_1,\theta_2,...,\theta_n) \in S_1\times S_2 \times...\times S_n$. The hyperparameters tested consist of: 




We follow the experimental setup from the PGExplainer as closely as possible. Since the textual description refers to the setup from GNNExplainer and is lacking in some aspects, we extract the missing information from the codebase. As the hyperparameters are unclear or not comprehensible for some tasks we also draw information from the configs of PYTORCH REIMPL. \\



Additionally, experiment in the collective setting with the configuration and setup described for inductive setting - difference lies test data seen during training. Used for fair comparison with the original paper and the replication paper.

\begin{table}[ht]
    \centering
    \scriptsize
    \begin{tabularx}{\textwidth}{l*{6}{X}}   % 'X' column type from tabularx automatically scales columns
    \toprule
    \textbf{} & \multicolumn{6}{c}{\textbf{Explanation AUC}} \\
    \cmidrule{2-7}
    \textbf{Method} & BA-Shapes & BA-Community & Tree-Cycles & Tree-Grid & BA-2Motif & MUTAG \\
    \midrule
    PGExplainer & 0.963$\pm$0.011 & 0.945$\pm$0.019 & 0.987$\pm$0.007 & 0.907$\pm$0.014 & 0.926$\pm$0.021 & 0.873$\pm$0.013 \\
    \midrule
    RE-PGExplainer & 0.999$\pm$0.000 & 0.825$\pm$0.040 & 0.760$\pm$0.014 & 0.679$\pm$0.008 & 0.133$\pm$0.046 & 0.843$\pm$0.084 \\
    \midrule
    PGExplainer (inductive) & $\sim$0.98 & $\sim$0.99 & $\sim$0.99 & $\sim$0.88 & $\sim$0.84 & - \\
    \midrule
    \textit{Inference Time PGExplainer (ms)} & 10.92 & 24.07 & 6.36 & 6.72 & 80.13 & 9.68 \\
    \textit{Inference Time RE-PGExplainer (ms)} & 3.58 & 5.23 & 0.45 & 0.54 & 0.33 & 2.05 \\
    \bottomrule
    \end{tabularx}
    \caption[Baseline PGExplainer and RE-PGExplainer]{PGExplainer performance baselines in collective setting.}
    \label{tab:pgexplainer_auc}
\end{table}

\begin{table}[ht]
    \centering
    \scriptsize
    \begin{tabularx}{\textwidth}{l*{4}{X}}   % Now only 4 datasets
    \toprule
    \textbf{} & \multicolumn{4}{c}{\textbf{Explanation AUC}} \\
    \cmidrule{2-5}
    \textbf{Method} & BA-Shapes & BA-Community & Tree-Cycles & Tree-Grid \\
    \midrule
    Our work (inductive) & 0.993$\pm$0.002 & 0.829$\pm$0.008 & 0.109$\pm$0.108 & 0.689$\pm$0.027 \\
    \addlinespace
    \midrule
    \midrule
    \textit{Inference Time (ms)} & 39.0$\pm$1.9 & 25.6$\pm$1.5 & 3.1$\pm$0.3 & 3.4$\pm$0.1 \\
    \bottomrule
    \end{tabularx}
    \caption[REMOVE! PGExplainer with $N=0.08$!]{PGExplainer performance WITH $N=0.08$! (4.8, 64, 4.8, 23) (USED IN SWEEP!)}
    \label{tab:pgexplainer_auc}
\end{table}

\begin{table}[ht]
    \centering
    \scriptsize
    \begin{tabularx}{\textwidth}{l*{6}{X}}   % 'X' column type from tabularx automatically scales columns
    \toprule
    \textbf{} & \multicolumn{6}{c}{\textbf{Explanation AUC}} \\
    \cmidrule{2-7}
    \textbf{Method} & BA-Shapes & BA-Community & Tree-Cycles & Tree-Grid & BA-2Motif & MUTAG \\
    \midrule
    PGExplainer (collective) & 0.963$\pm$0.011 & 0.945$\pm$0.019 & 0.987$\pm$0.007 & 0.907$\pm$0.014 & 0.926$\pm$0.021 & 0.873$\pm$0.013 \\
    \midrule
    RE-PGExplainer (collective) & 0.999$\pm$0.000 & 0.825$\pm$0.040 & 0.760$\pm$0.014 & 0.679$\pm$0.008 & 0.133$\pm$0.046 & 0.843$\pm$0.084 \\
    \midrule
    PGExplainer (inductive) & $\sim$0.98 & $\sim$0.99 & $\sim$0.99 & $\sim$0.88 & $\sim$0.84 & - \\
    \midrule
    Our work (inductive) & 0.994$\pm$0.001 & 0.754$\pm$0.013 & 0.106$\pm$0.104 & 0.537$\pm$0.081 & 0.017$\pm$0.006 & 0.874$\pm$0.009 \\
    \addlinespace
    \midrule
    \midrule
    \textit{Inference Time (ms)} & 37.0$\pm$1.4 & 24.8$\pm$0.1 & 3.0$\pm$0.2 & 2.7$\pm$0.1 & 4.0$\pm$0.3 & 4.0$\pm$0.0 \\
    \bottomrule
    \end{tabularx}
    \caption[Inductive performance of our reimplementation]{PGExplainer performance in Explanation AUROC compared to baselines ($N=30$).}
    \label{tab:pgexplainer_auc}
\end{table}

\textbf{Results}

HERE GO THE RESULTS OF THE EXPERIMENTS, ALSO COMPARED TO THE OTHER PAPERS. INCLUDES, TABLES, PLOTS ETC.

We use normalization in our downstream models, though it is not described in the paper, since it is used in the code. We also experimented with the effects of the use of normalization since it seems to be relevant to the performance of the explainer. \\
The explainer is trained and evaluated on the same data. We also run experiments with added train/test splits TODO! \\
Not all data is fed into the explainer:
BA-Shapes: 
BA-Community:
Tree-Cycles: "First"/to base graph attached node of each motif in graph is used.
Tree-Grid: All Motif nodes are used
BA-2Motif: All graphs are used
MUTAG: Only the graphs with an available ground truth are used. GT exist for mutagenic graphs that have either chemical groups NH2 or NO2.
We later discuss if these selections make sense and run experiments with different data selections.

Quantitative: 10 explainer runs on one downstream model; Calculate ROC-AUC over ALL graphs/nodes in each run;
Qualitative: Original uses a threshold; we instead take the topK nodes according to the dataset/motif as an explanation

We also discuss if treating the number of k edges as a hyperparameter dependant on the downstream task makes sense and propose having the network learn it, to improve generalizability and allow the explainer to work on data with varying size. \\

CPU vs GPU: \\
- Effects of using "wrong" node features for downstream task?

Comparison to orginal one: Original dataset transformed to pytorch performs way worse, for features of 0.1! Mean AUC of about 0.4! \\
Original dataset with features changed to ones instead of 0.1: Works good as well.
%It is important to highlight that our code achieved better and way more stable results for BA-2Motif when trained on a gpu instead of cpu. \\
%Ba-Shapes, Tree-Cycles and MUTAG results achieved were identical. \\
%Ba-Community and Tree-Grid achieved very slightly better results on CPU. \\

%Sweeps: (Params ordered by importance)\\
%BA-Shapes: higher size reg -> 0.1; lower entropy reg -> 0.01; lr and tT very low impact but slightly higher -> 0.01 and 5. Note that Loss curve jumps on most runs! (logical-sweep-94 and restful-sweep-92 have clean loss) TRY HIGHER SIZE REG AND LOWER ENTROPY REG \\

%BA-Community: lr 0.0001 too low, not working -> 0.003; lower entropy -> 0.1; higher size -> 0.1; TRY MORE SEEDS, LR, EPOCHS? \\

%Tree-Cycles: high lr -> 0.01 ; lower entropy reg -> 0.1/0.01; higher size reg -> 0.1/0.01 ; lower tT -> 1. TRY WITH MORE SEEDS FOR ENT, SIZE, TEMP? Confirmed higher size reg -> 0.1; lower entropy reg -> 0.01; temp really low impact, tendency higher. TRY 30 EPOCHS???\\

%Tree-Grid: high lr -> 0.01; high size reg -> 1; higher entropy reg? -> 10/1, high tT -> 5. TRY MORE SEEDS FOR ENTROPY REG -> not quite clear, tendency lower; MAYBE EVEN HIGHER LR -> No\\

%BA-2Motif: RUN ON GPU - Not the cause. Cause for better results were features of 1 instead of 0.1! However, good results achieved on BA2-Motif dataset from pyg, not original one.\\

%MUTAG: Low lr -> 0.0003; low entropy reg(high impact, but highest AUC runs vary) -> 0.1; low tT -> 1; less epochs -> 20; low size reg -> 0.005(/0.01); Loss is messy and AUC seems to decrease over time! lr 0.0001 worse, entropy reg 0.1/0.01 has zero effect -> 0.1\\


RESULTS OBTAINED IN COLLECTIVE SETTING (ORIGINAL PAPER):

\begin{table}[h]
    \centering
    \scriptsize
    \begin{tabularx}{\linewidth}{l|X X X X|X X}
    \hline
    \textbf{Accuracy} & \textbf{BA-Shapes} & \textbf{BA-Community} & \textbf{Tree-Cycles} & \textbf{Tree-Grid} & \textbf{BA-2Motif} & \textbf{MUTAG} \\
    \hline
    \textbf{Training}   & 1.00 & 0.92 & 1.00 & 0.99 & 1.00 & 0.86 \\
    \textbf{Validation} & 1.00 & 0.87 & 1.00 & 0.99 & 1.00 & 0.86 \\
    \textbf{Testing}    & 0.97 & 0.89 & 0.99 & 0.99 & 1.00 & 0.82 \\
    \hline
    \end{tabularx}
    \caption[Accuracies of higher-order GNN downstream task]{Compact accuracy table for Node and Graph Classification from our reimplementation.}
    \label{tab:compact-accuracy}
\end{table}


\begin{table}[h]
    \centering
    \scriptsize
    \begin{tabularx}{\linewidth}{l|X X X X|X X}
    \hline
    \textbf{Accuracy} & \textbf{BA-Shapes} & \textbf{BA-Community} & \textbf{Tree-Cycles} & \textbf{Tree-Grid} & \textbf{BA-2Motif} & \textbf{MUTAG} \\
    \hline
    \textbf{Training}   & 0.98 & 0.99 & 0.99 & 0.92 & 1.00 & 0.87 \\
    \textbf{Validation} & 1.00 & 0.88 & 1.00 & 0.94 & 1.00 & 0.89 \\
    \textbf{Testing}    & 0.97 & 0.93 & 0.99 & 0.94 & 1.00 & 0.87 \\
    \hline
    \end{tabularx}
    \caption[Accuracies of original GNN downstream task]{Compact accuracy table for Node and Graph Classification from ORIGINAL PGExplainer paper.}
    \label{tab:compact-accuracy}
\end{table}


\subsection{PGExplainer in inductive setting with more training data e.g. 60 training instances}

\textbf{Experimental setup}

\textbf{Results}


BA-Community: SumSampledEdges increases after 6/7 epochs, AUROC reaches peak early but then decreases steadily! (For 30 instances: Both sum sampled egdes and AUROC converge very strongly!) - Apparently less stable on 60 instances (ONLY CHANGE!!!), probably would have to be retuned (BAD!)
\begin{table}[ht]
    \centering
    \scriptsize
    \begin{tabularx}{\textwidth}{l*{6}{X}}   % 'X' column type from tabularx automatically scales columns
    \toprule
    \textbf{} & \multicolumn{6}{c}{\textbf{Explanation AUC}} \\
    \cmidrule{2-7}
    \textbf{Method} & BA-Shapes & BA-Community & Tree-Cycles & Tree-Grid & BA-2Motif & MUTAG \\
    \midrule
    Our work (inductive) & 0.994$\pm$0.001 & 0.754$\pm$0.013 & 0.106$\pm$0.104 & 0.537$\pm$0.081 & 0.017$\pm$0.006 & 0.874$\pm$0.009 \\
    \midrule
    60 train instance & 0.987.$\pm$0.002 & 0.641$\pm$0.045 & 0.144$\pm$0.096 & 0.495$\pm$0.052 & 0.017$\pm$0.004 & 0.895$\pm$0.009 \\
    \bottomrule
    \end{tabularx}
    \caption[Inductive performance with 60 training instances]{PGExplainer performance in Explanation AUROC ($N=60$).}
    \label{tab:pgexplainer_auc}
\end{table}


\subsection{PGExplainer in collective setting}

\textbf{Experimental setup}
Collective setting (ALL instances used for training + testing) with hyperparameters that were found in inductive setting.

\textbf{Results}

\begin{table}[ht]
    \centering
    \scriptsize
    \begin{tabularx}{\textwidth}{l*{6}{X}}   % 'X' column type from tabularx automatically scales columns
    \toprule
    \textbf{} & \multicolumn{6}{c}{\textbf{Explanation AUC}} \\
    \cmidrule{2-7}
    \textbf{Method} & BA-Shapes & BA-Community & Tree-Cycles & Tree-Grid & BA-2Motif & MUTAG \\
    \midrule
    PGExplainer (collective) & 0.963$\pm$0.011 & 0.945$\pm$0.019 & 0.987$\pm$0.007 & 0.907$\pm$0.014 & 0.926$\pm$0.021 & 0.873$\pm$0.013 \\
    \midrule
    RE-PGExplainer (collective) & 0.999$\pm$0.000 & 0.825$\pm$0.040 & 0.760$\pm$0.014 & 0.679$\pm$0.008 & 0.133$\pm$0.046 & 0.843$\pm$0.084 \\
    \midrule
    Our work (collective) & 0.993$\pm$0.001 & 0.831$\pm$0.009 & 0.084$\pm$0.078 & 0.679$\pm$0.032 & 0.018$\pm$0.004 & 0.833$\pm$0.006 \\
    \midrule
    \midrule
    Our work (inductive) & 0.994$\pm$0.001 & 0.754$\pm$0.013 & 0.106$\pm$0.104 & 0.537$\pm$0.081 & 0.017$\pm$0.006 & 0.874$\pm$0.009 \\
    \bottomrule
    \end{tabularx}
    \caption[Collective performance of our reimplementation]{PGExplainer performance in Explanation AUROC in collective setting.}
    \label{tab:pgexplainer_auc}
\end{table}

\subsection{BA-2Motif with flipped GT}

\textbf{Experimental setup}
Mean individual AUROC is calculated identical to before, but the ground truth masks of each graph are inverted, meaning that edges in the motif now carry a label of $0$ and all other edges a label of $1$. Validation loss is evaluated as well.

\textbf{Results}

\begin{table}[ht]
    \centering
    \scriptsize
    \begin{tabularx}{0.4\textwidth}{l X}
        \toprule
        \textbf{Method} & \textbf{Explanation AUC (BA-2Motif)} \\
        \midrule
        Our work       & 0.017 $\pm$ 0.006 \\
        Flipped GT     & 0.985 $\pm$ 0.006 \\
        \bottomrule
    \end{tabularx}
    \caption[Inductive performance on BA-2Motif with flipped ground truth]{Explanation AUROC for BA-2Motif with flipped GT.}
    \label{tab:allmotifnodes_selected}
\end{table}

Validation Loss curve relatively smooth. TODO: INCLUDE

\subsection{Effects of motif node sets}

\textbf{Experimental setup}

AllMotifNodes experiment:
BA-Shapes: All house nodes
Tree-Cycles: All Cycle nodes
\bigskip


OneMotifNode experiment:

BA-Community:
%motifNodes = [i for i in range(single_label.shape[0]) if single_label[i] == 1 and single_label[i] != 5]
(One of the two middel house nodes)

Tree-Grid:
[512,800,9]
(One of the two nodes that connect to the base connection node, to contain full motif)

Maybe works well because if all motif nodes are used, subgraphs are used where only the motif or even only part of the motif is present (Tree-Grid). Thus constantly using one singular same node that contains the complete motif may be better?


\textbf{Results}

\begin{table}[ht]
    \centering
    \scriptsize
    \begin{tabularx}{0.6\textwidth}{l*{2}{X}}   % Adjust width as needed
    \toprule
    \textbf{} & \multicolumn{2}{c}{\textbf{Explanation AUC}} \\
    \cmidrule{2-3}
    \textbf{Method} & BA-Shapes & Tree-Cycles \\
    \midrule
    Our work & 0.994$\pm$0.001 & 0.106$\pm$0.104 \\
    \midrule
    AllMotifNodes & 0.959$\pm$0.004 & 0.204$\pm$0.162 \\
    \bottomrule
    \end{tabularx}
    \caption[Inductive performance using all motif nodes for training]{Explanation AUROC and inference time for AllMotifNodes on selected datasets.}
    \label{tab:allmotifnodes_selected}
\end{table}

\begin{table}[ht]
    \centering
    \scriptsize
    \begin{tabularx}{0.6\textwidth}{l*{2}{X}}   % Adjust width as needed
    \toprule
    \textbf{} & \multicolumn{2}{c}{\textbf{Explanation AUC}} \\
    \cmidrule{2-3}
    \textbf{Method} & BA-Community & Tree-Grid \\
    \midrule
    Our work & 0.754$\pm$0.013 & 0.537$\pm$0.081 \\
    \midrule
    OneMotifNode & 0.915$\pm$0.003 & 1.0$\pm$0.0 \\
    \bottomrule
    \end{tabularx}
    \caption[Inductive performance using one motif node for training]{Explanation AUROC and inference time for OneMotifNode on selected datasets.}
    \label{tab:allmotifnodes_selected}
\end{table}


\subsection{Qualitative Analysis}

\textbf{Experimental setup}

\textbf{Results}


%\subsection{Training on ALL instances - Evaluation only on motif instances?}
%This probably would not grant any insights, since training on non motif nodes should in theory mostly lead to all edges being irrelevant for prediction?? Bascially most edge could be removed and in theory the prediction of the node should not change at all, therefore no information can be gather from this?

%\textbf{Experimental setup}

%\textbf{Results}



\section{PGExplainer applied to NeuroSAT}

TODO: We present our approach for generating bipartite explanations for NeuroSAT \cite{} predictions of SAT problems.

\subsection{Common experimental setup}

We create required data with provided methods, add unsat cores and MUSs as gt


Generated batches of unsat problems that "turned" unsat because of last added clause. 10 literals per problem. Only unsat to test for unsat cores, that only apply for unsat problems. Calculated unsat cores with solver xy by adding negative assumption literals per clause and passing these as assumption for calulation. The edges of the clauses present in the unsat core were treated as ground truth. \\


For quant. eval. adapted roc auc as metric as done in PGExplainer. Results seem "good" but qual. eval. shows different result. roc auc bad metric? \\
For qual. eval. topk(=number of edges in gt) edges of predictions were highlighted to be compared to gt edges. For quant. eval. the edge probabilites were compared to gt with 1s for edges in gt and 0s for rest. \\

\textbf{Hyperparameter search}
We perform a hyperparameter search for both the explainer with the soft constraint, and the explainer with the hard constraints.

\textbf{Results}
