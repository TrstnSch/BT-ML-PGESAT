\chapter{PGExplainer - Methodology}
\label{ch:PGExplainer}
%V1: In the following chapter, we introduce the PGExplainer\cite{luo2020parameterized} and its concepts. The idea is to generate explanations in the form of probabilistic graph generative models for any learned GNN model, henceforth referred to as the downstream task (DT), by utilizing a deep neural network to parameterize the generation process. This approach seeks to explain multiple instances collectively, as they share the same neural network parameters, and therefore improve the generalizability to previous works, particularly the GNNExplainer\cite{ying2019gnnexplainer}.

%V2: In the following chapter, we introduce the PGExplainer\cite{luo2020parameterized} and its concepts. The idea is to generate explanations in the form of probabilistic graph generative models that have proven to learn the concise underlying structures of GNNs most relevant to their predictions. This approach may be applied to any learned GNN model, henceforth referred to as the downstream task (DT), by utilizing a deep neural network to parameterize the generation process. PGExplainer seeks to explain multiple instances collectively, as they share the same neural network parameters, and therefore improve the generalizability to previous works, particularly the GNNExplainer\cite{ying2019gnnexplainer}. This means that all edges in the dataset are predicted by the same model, which leads to a global understanding of the DT.

In this chapter we present the Methodology (TODO: ?) of our work, as well as the detailed theory of the explainer model used in our approach. Therefore, we start by presenting the concept of the Parameterized explainer for graph neural networks by Luo et al. \cite{luo2020parameterized} in Section \ref{sec:TheoryPGE}.

In Section \ref{} we present our idea of applying the PGExplainer on the NeuroSAT framework to generate explanations for a machine learning SAT-solving approach and comparing these to "human-understandable" concepts like UNSAT cores. 

We then describe our imlpementation in detail (see Section \ref{}), including the changes made and difficulties during the process.

In Section \ref{} we describe the experimental setups for both the replication of the PGExplainer in the inductive setting, as well as the application on NeuroSAT.

\section{Theoretical Foundations of PGExplainer}
\label{sec:TheoryPGE}

In the following subchapter, we introduce the PGExplainer by Luo et al. \cite{luo2020parameterized} and its concepts. The idea is to generate explanations in the form of edge distributions or soft masks using a probabilistic generative model for graph data, known for being able to learn the concise underlying structures from the observed graph data. The explainer uncovers said underlying structures, believed to have the biggest impact on the prediction of a GNNs, as explanations. This approach may be applied to any trained GNN model, henceforth referred to as the target model (TM). 
By utilizing a deep neural network to parameterize the generation process, the explainer learns to collectively explain multiple instances of a model. Since the parameters of the neural network are shared across the population of explained instance, PGExplainer provides "model-level explanations for each instance with a global view of the GNN model" \cite{luo2020parameterized}. Furthermore, this approach cannot only be used in a collective setting, but also in an inductive setting, where explanations for unexplained nodes can be generated without retraining the explanation model. This improves the generalizability compared to previous works, particularly the GNNExplainer by Ying et al. \cite{ying2019gnnexplainer}.


TODO: The focus in this approach lies in explaining the graph structure, rather than the graph features, as feature explanations are already common in non-graph neural networks. \bigskip

We follow the structure of the original paper \cite{luo2020parameterized} and start by describing the learning objective in Section \ref{sec:learning_objective}, the utilized reparameterization trick in Section \ref{sec:Reparameterization_Trick}, the idea of global explanations in Section \ref{sec:Global_Explanations} and finally the applied regularization terms in Section \ref{sec:Regularization_Terms}.

\subsection{Learning Objective}
\label{sec:learning_objective}
To explain the predictions made by a GNN model for an original input graph $G_o$ with $m$ edges we first define the graph as a combination of two subgraphs: $G_o = G_s + \Delta G$, where $G_s$ represents the subgraph holding the most relevant information for the prediction of a GNN, referred to as explanatory graph. $\Delta G$ contains the remaining edges that are deemed irrelevant for the prediction of the GNN. Inspired by GNNexplainer \cite{ying2019gnnexplainer}, the PGExplainer then finds $G_s$ by maximizing the mutual information between the predictions of the target model and the underlying $G_s$:
\begin{equation}
    \max_{G_s} MI(Y_o;G_s) = H(Y_o) - H(Y_o|G=G_s),
\end{equation} 
where $Y_o$ (TODO: $\in (0,1)^c$ ??) is the prediction of the target model with $G_o$ as input and number of possible classes $c$. This quantifies the probability of prediction $Y_o$ when the input graph is restricted to the explanatory graph $G_s$, as in the case of $I(Y_o;G_s) = 1$, knowing the explanatory graph $G_s$ gives us complete information about $Y_o$, and vice versa. Intuitively, if removing an edge $(i,j)$ changes the prediction of a GNN drastically, this edge is considered important and should therefore be included in $G_s$. This idea originates from traditional forward propagation based methods for whitebox explanations (see Dabkowski et al. \cite{dabkowski2017real}).
It is important to note that $H(Y_o)$ is only related to the target model with fixed parameters during the evaluation/explanation stage. This leads to the objective being equivalent to minimizing the conditional entropy $H(Y_o|G=G_s)$.

To optimize this function a relaxation is applied for the edges, since normally there would be $2^m$ candidates for $G_s$. The explanatory graph is henceforth assumed to be a Gilbert random graph, where the selections of edges from $G_o$ are conditionally independent to each other. However, the authors describe a random graph with each edge having its own probability, rather than a shared probability as described in Section \ref{sec:random-graphs}, as follows: Let $e_{ij}\in V \times V$ be the binary variable indicating whether the edge is selected, with $e_{ij} = 1$ if edge $(i,j)$ is selected to be in the graph, and 0 otherwise. For the random graph variable $G$ the probability of a graph $G$ can be factorized as 
\begin{equation}
    P(G) = \prod_{(i,j)\in E}P(e_{ij}).
\end{equation}
TODO: Inhomogeneous Erdos Renyi model? Mention that this is a generative model? (A Gilbert random graph is an example of a generative probabilistic model on graph data?)

 $P(e_{ij})$ is instantiated with the Bernoulli distribution $e_{ij} \sim Bern(\theta_{ij})$, where $P(e_{ij} = 1) = \theta_{ij}$ is the probability that edge $(i,j)$ exists in $G$.
After this relaxation the learning objective becomes:
\begin{equation}
    \label{eq:init_learning_obj}
    \min_{G_s}H(Y_o|G = G_s) = \min_{G_s} \mathbb{E}_{G_s}[H(Y_o|G = G_s)] \approx \min_{\Theta} \mathbb{E}_{G_s \sim q(\Theta)}[H(Y_o|G = G_s)],
\end{equation}
where $q(\Theta)$ is the distribution of the explanatory graph that is parameterized by $\Theta$'s.

\subsection{Reparameterization Trick}
\label{sec:Reparameterization_Trick}
As described in Section \ref{sec:gnn_explainability}, a reparameterization trick can be utilized to relax discrete edge weights to continuous variables in the range $(0,1)$. PGExplainer uses the reparameterizable Gumbel-Softmax estimator \cite{jang2016categorical} to allow for efficiently optimizing the objective function with gradient-based methods. This method introduces the Gumbel-Softmax distribution, a continuous distribution used to approximate samples from a categorical distribution. A temperature $\tau$ is used to control the approximation, usually starting from a high value and annealing to a small, non-zero value. Samples with $\tau > 0$ are not identical to samples from the corresponding continuous distribution, but are differentiable and therefore allow back-propagation \cite{abid2019concrete}. The sampling process $G_s \sim q(\Theta)$ of PGExplainer is therefore approximated with a determinant function that takes as input the parameters $\Omega$, a temperature $\tau$ and an independent random variable $\epsilon$: $G_s \approx \hat{G}=f_\Omega(G_o,\tau,\epsilon)$. The binary concrete distribution \cite{maddison2016concrete}, also referred to as Gumbel-Softmax distribution, is utilized as an instantiation for the sampling, yielding the weight $\hat{e}_{ij} \in (0,1)$ for edge $(i,j)$ in $\hat{G}_s$, computed by:
\begin{equation}
    \label{eq:reparam_trick}
    \epsilon \sim \text{Uniform}(0,1), \qquad \hat{e}_{ij}=\sigma((\log \epsilon - \log(1-\epsilon)+\omega_{ij}/\tau),
\end{equation}
where $\sigma(\cdot)$ is the Sigmoid function and $\omega_{ij} \in \mathbb{R}$ is an explainer logit for the corresponding edge used as a parameter. When $\tau \rightarrow 0$, e.g. during the explanation stage, the weight $\hat{e}_{ij}$ is binarized with the sigmoid function $\lim_{\tau\rightarrow 0}P(\hat{e}_{ij} = 1) = \frac{\exp (\omega{ij})}{1+\exp (\omega{ij})}$. Since $P(e_{ij} = 1) = \theta_{ij}$, choosing $\omega_{ij} = \log\frac{\theta_{ij}}{1-\theta_{ij}}$ leads to $\lim_{\tau\rightarrow 0}\hat{G}_s = G_s$ and justifies the approximation of the Bernoulli distribution with the binary concrete distribution. During training, when $\tau > 0$, the objective function in \eqref{eq:init_learning_obj} is smoothed with a well-defined gradient $\frac{\partial\hat{e}_{ij}}{\partial\omega_{ij}}$ and becomes:
\begin{equation}
    \min_\Omega \mathbb{E}_{\epsilon \sim \text{Uniform}(0,1)}H(Y_o| G = \hat{G}_s)
\end{equation}

The authors follow the approach of GNNExplainer \cite{ying2019gnnexplainer} and modify the objective by replacing the conditional entropy with cross entropy between the label class and the prediction of the target model. This is justified by the greater importance of understanding the model's prediction of a certain class, rather than providing an explanation based solely on its confidence.

With the modification to cross-entropy $H(Y_o, \hat{Y}_s)$, where $\hat{Y}_s$ is the prediction of the target model when $\hat{G}_s$ is given as input, as well as the adaption of Monte Carlo sampling, the learning objective becomes:
\begin{equation}
    \label{eq:monte_carlo}
    \begin{aligned}
        \min_\Omega\mathbb{E}_{\epsilon\sim\text{Uniform}(0,1)}H(Y_o, \hat{Y}_s) &\approx \min_\Omega -\frac{1}{K}\sum_{k=1}^K\sum_{c=1}^C P(Y_o = c) \log P(\hat{Y}_s = c) \\
        &= \min_\Omega -\frac{1}{K}\sum_{k=1}^K\sum_{c=1}^C P_\Phi (Y_o = c|G = G_o) \log P_\Phi(\hat{Y}_s = c|G=\hat{G}_s^{(k)}).
    \end{aligned}
    \end{equation}
$\Phi$ denotes the parameters in the target model, $K$ is the number of total sampled graphs, $C$ is the number of class labels, and $\hat{G}_s^{(k)}$ denotes the $k$-th graph sampled with Equation \ref{eq:reparam_trick}, parameterized by $\Omega$. %Note that this objective is defined per explainable instance.



%The approach in PGExplainer is a common approach in ML for simplifying objectives? FIND LITERATURE THAT EXPLAINS APPROXIMATION OF COND. ENTROPY WITH CROSS ENTROPY. Explanation as simple as one formula for one graph variable example, cross entropy applied to whole distribution? \bigskip


\subsection{Global Explanations}
\label{sec:Global_Explanations}
The novelty of PGExplainer lies in the ability to generate explanations for graph data with a global perspective, that allow for understanding the general picture of a model across a population. This saves resources when analyzing large graph datasets, as new instances can be explained without retraining the model, and can also be helpful for establishing the users' trust in these explanations. To achieve this the authors propose the use of a parameterized network that learns to generate explanations from the target model, which also apply to not yet explained instances. 

Since GNNs apply two functions $F$ and $G$ to calculate the global state embeddings and downstream task outputs respectively, we denote these two functions as $\text{GNNE}_{\Phi_0}(\cdot)$ and $\text{GNNC}_{\Phi_1}(\cdot)$ for any GNN in the context of PGExplainer. For models without explicit classification layers the last layer is used to compute the output instead. It follows
\begin{equation}
    \mathbf{Z} = \text{GNNE}_{\Phi_0}(G_o, \mathbf{X}), \qquad Y = \text{GNNC}_{\Phi_1}(\mathbf{Z}),
\end{equation}
where $\mathbf{Z}$ denotes the matrix of final node representations $z$, referred to as node embeddings, and the initial state is $G_o$. TODO: (Because of focus on graph structure rather than features?) For generalizability across different GNN layers the output is only dependent on the node representation, that encapsulates both features and structure of the input graph. This representation also serves as the input for the explainer network $g$, defined as:
TODO: Rename g?
\begin{equation}
    \label{eq:explainer_network}
    \Omega = g_\Psi(G_o,\mathbf{Z}).
\end{equation}
$\Psi$ denotes the parameters in the explanation network and the output $\Omega$ is treated as parameter in Equation \ref{eq:monte_carlo}. Since $\Psi$ is shared by all edges among the population, PGExplainer collectively provides explanations for multiple instances. Thus, the learning objective in a collective setting with $\mathcal{I}$ being the set of instances becomes:
\begin{equation}
    \label{eq:mlp_loss}
    \min_\Psi -\frac{1}{K}\sum_{i\in \mathcal{I}}\sum_{k=1}^K\sum_{c=1}^C P_\Phi (Y_o = c|G = G_o^{(i)}) \log P_\Phi(\hat{Y}_s = c|G=\hat{G}_s^{(i,k)}).
\end{equation}
Consequently, $G^{(i)}$ and $G_s^{(i,k)}$ denote the input graph and the $k$-th graph sampled with Equation \ref{eq:reparam_trick} in \ref{eq:explainer_network} respectively for instance $i$. The full pipeline of the PGExplainer can be observed in figure \ref{fig:PGExplainer_pipeline}. 

The authors propose two slightly different instantiations of $\Omega$ for node classification and graph classification tasks.

\textbf{Explanation network for graph classification}

For graph level tasks, the authors consider each graph to be an instance, independent  of specific nodes. The output $\Omega$ of the network (see Equation \ref{eq:explainer_network}) is thus specified as:
\begin{equation}
    \label{eq:mlp_graph_input}
    \omega_{ij} = \text{MLP}_\Psi ([\mathbf{z}_i\oplus\mathbf{z}_j]),
\end{equation}
where $\text{MLP}_\Psi$ is an MLP ( see \ref{} for implementation details) parameterized with $\Psi$ and $\oplus$ denotes the concatenation operation. Effectively, for each edge in $G_o$ a concatenation of both its nodes is fed through the MLP. The output $\omega_{ij}$ is therefore an edge logit (TODO: NOT IN GENERAL? ONLY FOR THE LATER SPECIFIED MLP?), which serves as a parameter in the sampling process.

\textbf{Explanation network for node classification}
%Since explanations for different nodes may not share a common explanation pattern, especially for nodes with different labels, ... DOES THIS NOT ALSO APPLY FOR GRAPHS?!

For node level tasks on the other hand, each prediction node is considered as an instance. Let an edge $(i,j)$ be considered relevant for the prediction of a node $u$, but irrelevant for the prediction of a node $v$. To explain the prediction of node $v$ the authors specify the output $\Omega$ of the network in Equation \ref{eq:explainer_network} as:
\begin{equation}
    \omega_{ij} = \text{MLP}_\Psi ([\mathbf{z}_i\oplus\mathbf{z}_j\oplus\mathbf{z}_v]).
\end{equation}
Thus, a concatenation of the node embeddings of nodes $i, j$ and $v$ respectively is fed through the network.

Note that in their codebase the authors use a concatenation of all hidden representations instead of solely the final representation as node embeddings for node level tasks.
For a target GNN consisting of $L$ graph layers with
\begin{align*}
    \mathbf{H}_1 &= F_1(G_o, \mathbf{X}), \\
    \mathbf{H}_2 &= F_2(\mathbf{H}_1, \mathbf{X}), \\
    &\vdots \\
    \mathbf{H}_L &= F_L(\mathbf{H}_{L-1}, \mathbf{X}), \\
\end{align*}
this leads to $\mathbf{Z} \in \mathbb{R}^{V(G_o)\times (Ld)}$ being the matrix of node embeddings $\mathbf{z}$ that are computed as:
\begin{equation}
    \mathbf{z}_i = \mathbf{h}_{1,i} \oplus \mathbf{h}_{2,i} \oplus ... \oplus \mathbf{h}_{L,i},
\end{equation}
with $\mathbf{h}_{L,i}$ denoting the hidden representations of node $i$ in layer $L$\bigskip

Furthermore, it is important to note, that for target GNNs utilizing a message passing mechanism, the prediction at node $v$ is fully determined by its local computation graph. This is defined by its $L-$hop neighborhood $\mathcal{N}_L(v)$ \cite{ying2019gnnexplainer}. TODO: SEE FIGURE\bigskip

\textbf{Collective and inductive setting}

Due to the nature of its predecessor GNNExplainer, the authors main focus was the application in a collective setting, where the goal is to explain a full population of instances by training on all these and thus being able to provide explanations for every single one. However, since the PGExplainer utilizes a deep neural network to parameterize the generation process of explanations for a population, it can be utilized in an inductive setting, unlike its predecessor. This means, that explanations can be generated for instances from the same population, that have not been seen during training. Thus, it is not necessary to retrain the explainer for new instances of the same population, effectively reducing the computational complexity by the training time when compared to the GNNExplainer.

We discuss the results of the inductive performance study by the authors in \ref{}.


%"PGExplainer learns a latent variable for each edge in the original input graph with a neural network parameterized by Ψ, which is shared by all edges in the population of input graphs." "In GNNExplainer, the parameter size is linear to the number of edges"

This leads to an improved computational complexity when compared to their baseline GNNExplainer, since for one the number of parameters in the explainer does no longer depend on the size of the input graph and since for another the explainer does not have to be retrained for every unexplained instance.

\tikzstyle{process} = [rectangle, rounded corners, minimum width=3cm, minimum height=1cm, text centered, draw=black, fill=purple!30, font=\small]
\tikzstyle{module} = [rectangle, rounded corners, minimum width=3cm, minimum height=1.2cm, text centered, draw=black, fill=gray!30, font=\small]
\tikzstyle{data} = [rectangle, sharp corners, minimum width=1.5cm, minimum height=1cm, text centered, draw=black, fill=cyan!30, font=\small]
\tikzstyle{emb} = [rectangle, sharp corners, minimum width=1.5cm, minimum height=1cm, text centered, draw=black, fill=yellow!30, font=\small]
\tikzstyle{arrow} = [very thick,->,>=Stealth]    %very thick

\begin{figure}
\centering
\begin{tikzpicture}[node distance=0.9cm and 1.2cm]

\tikzset{
  mymini/.pic={
    \node[circle, draw, fill=black, inner sep=2pt, label=left:$x_i$] (x) at (0,0) {};
    \node[circle, draw, fill=black, inner sep=2pt, label=above:$x_j$] (y) at (0.5,0.5) {};
    \node[circle, draw, fill=black, inner sep=2pt, label=left:$x_k$] (a) at (0,0.5) {};
    \node[circle, draw, fill=black, inner sep=2pt, label=left:$x_l$] (b) at (0,-0.5 ) {};
    \draw[-] (x) -- (y);
    \draw[-] (a) -- (y);
    \draw[-] (b) -- (x);
    \draw[-] (a) -- (x);
  }
}

\tikzset{
  mymini3/.pic={
    \node[circle, draw, fill=black, minimum size=4pt, inner sep=2pt, label=left:$z_i$] (x) at (0,0) {};
    \node[circle, draw, fill=black, minimum size=4pt, inner sep=2pt, label=above:$z_j$] (y) at (0.5,0.5) {};
    \node[circle, draw, fill=black, minimum size=4pt, inner sep=2pt, label=left:$z_k$] (a) at (0,0.5) {};
    \node[circle, draw, fill=black, minimum size=4pt, inner sep=2pt, label=left:$z_l$] (b) at (0,-0.5) {};
    \draw[-] (x) -- (y);
    \draw[-] (a) -- (y);
    \draw[-] (b) -- (x);
    \draw[-] (a) -- (x);
  }
}

\tikzset{
  mymini2/.pic={
    \node[circle, draw, fill=black, inner sep=2pt] (x) at (0,0) {};
    \node[circle, draw, fill=black, inner sep=2pt] (y) at (0.5,0.5) {};
    \node[circle, draw, fill=black, inner sep=2pt] (a) at (0,0.5) {};
    \node[circle, draw, fill=black, inner sep=2pt] (b) at (0,-0.5 ) {};
    \draw[-] (x) -- node[midway, right, font=\scriptsize] {$0.9$} (y);
    \draw[-] (a) -- node[midway, above, font=\scriptsize] {$0.8$}(y);
    \draw[-] (b) -- node[midway, left, font=\scriptsize] {$0.1$}(x);
    \draw[-] (a) -- node[midway, left, font=\scriptsize] {$0.9$}(x);
  }
}

\tikzset{
  mymini4/.pic={
    \node[circle, draw, fill=black, inner sep=2pt] (x) at (0,0) {};
    \node[circle, draw, fill=black, inner sep=2pt] (y) at (0.5,0.5) {};
    \node[circle, draw, fill=black, inner sep=2pt] (a) at (0,0.5) {};
    \node[circle, draw, fill=black, inner sep=2pt] (b) at (0,-0.5 ) {};
    \draw[-] (x) -- node[midway, right, font=\scriptsize] {$z_{ij}$} (y);
    \draw[-] (a) -- node[midway, above, font=\scriptsize] {$z_{jk}$}(y);
    \draw[-] (b) -- node[midway, left, font=\scriptsize] {$z_{il}$}(x);
    \draw[-] (a) -- node[midway, left, font=\scriptsize] {$z_{ik}$}(x);
  }
}

\tikzset{
  mymini5/.pic={
    \node[circle, draw, fill=black, inner sep=2pt] (x) at (0,0) {};
    \node[circle, draw, fill=black, inner sep=2pt] (y) at (0.5,0.5) {};
    \node[circle, draw, fill=black, inner sep=2pt] (a) at (0,0.5) {};
    \node[circle, draw, fill=black, inner sep=2pt] (b) at (0,-0.5 ) {};
    \draw[-] (x) -- (y);
    \draw[-] (a) -- (y);
    \draw[-] (a) -- (x);
  }
}


% Nodes
\node (input) [data] {\small Input Graph $G_o$};
\pic at ([xshift=7cm]input.center) {mymini};
\node (target) [module, below=of input] {Target GNN};
\node (embeddings) [emb, below=of target] {Node Embeddings $\mathbf{Z}$ of $G_o$};
\node[anchor=center] at ([xshift=4cm]embeddings.center) 
 (node_embs) 
 {$\begin{bmatrix} z_i \\ z_j \\ \vdots \end{bmatrix}$};
\node[right=0.01cm of node_embs, anchor=west] 
 {\scriptsize $[|V(G_o)|]$};
\pic at ([xshift=7cm]embeddings.center) {mymini3};
\node (embedding_transformation) [process, below=of embeddings] {Edge Embedding Transformation};
\node (edge_embeddings) [emb, below=of embedding_transformation] {Edge Embeddings of $G_o$};
\node[anchor=center] at ([xshift=4cm]edge_embeddings.center) 
 (edge_embs) 
 {$\begin{bmatrix} z_{ij} \\ z_{jk} \\ \vdots \end{bmatrix}$};
\node[right=0.01cm of edge_embs, anchor=west] 
 {\scriptsize $[|E(G_o)|]$};
\pic at ([xshift=7cm]edge_embeddings.center) {mymini4};
\node (explainer) [module, below=of edge_embeddings] {PGExplainer MLP};
\node (logits) [emb, below=of explainer] {Edge Logits/Latent variables $\Omega$};
\node[anchor=center] at ([xshift=4cm]logits.center) 
 (omega) 
 {$\begin{bmatrix} \omega_{ij} \\ \omega_{jk} \\ \vdots \end{bmatrix}$};
\node[right=0.01cm of omega, anchor=west] 
 {\scriptsize $[|E(G_o)|]$};
\node (trick) [process, below=of logits] {Reparameterization Trick};
\node (weights) [emb, below=of trick] {Sampled edge importance weights};
\node[anchor=center] at ([xshift=4cm]weights.center) 
 (edge_score) 
 {$\begin{bmatrix} \hat{e}_{ij} \\ \hat{e}_{jk} \\ \vdots \end{bmatrix}$};
\node[right=0.01cm of edge_score, anchor=west] 
 {\scriptsize $[|E(G_o)|]$};

\node (sampled_graph) [data, below= of weights] {Sampled graph $\hat{G}_s$};
\pic at ([xshift=7cm]sampled_graph.center) {mymini2};

\node (sample_target) [module, below=of sampled_graph] {Target GNN};
\node (original_target) [module, left=2cm of sample_target] {Target GNN};
\node (sample_prediction) [data, below=of sample_target] {$\hat{Y}_s$};
\pic at ([xshift=7cm]sample_prediction.center) {mymini5};
\node (original_prediction) [data, below=of original_target] {$Y_o$};

\node (topK) [process, right=of sample_target] {Sample top-$k$ edges};
\node (explanation) [data, below=of topK] {$G_s$};

%\node[draw=red, thick, dashed, fit=(input) (target) (embeddings), label=above:input Block] {};

%\begin{pgfonlayer}{background}
%    \node[draw=gray, thick, rounded corners, fit=(edge_embeddings) (weights), fill=blue!10, label=above:Sampling of PGExplainer] {};
%\end{pgfonlayer}

\begin{pgfonlayer}{background}
    \node[draw=gray, thick, rounded corners, fit=(original_target) (sample_target) (original_prediction) (sample_prediction), fill=orange!30] (backgroundtraining) {};
    \node[
        rotate=90, 
        anchor=center
    ] at ([xshift=-3mm]backgroundtraining.west) {Training};
\end{pgfonlayer}

\begin{pgfonlayer}{background}
    \node[draw=gray, thick, rounded corners, fit=(topK) (explanation), fill=green!30] (backgroundevaluation) {};
    \node[
        rotate=90, 
        anchor=center
    ] at ([xshift=-3mm]backgroundevaluation.west) {Evaluation};
\end{pgfonlayer}

\coordinate (weight_sample_mid) at ($ (weights)!0.5!(sampled_graph) $);
\coordinate (left_of_input) at ($ (input) + (-3.5cm, 0) $);
\coordinate (right_of_input) at ($ (original_target |- input) $);
\coordinate (drop) at (left_of_input |- weight_sample_mid);
\coordinate (right_of_sampled_graph) at ($ (topK |- sampled_graph) $);

% Arrows
\draw [arrow] (input) -- (target);
\draw [arrow] (target) -- (embeddings);
\draw [arrow] (embeddings) -- (embedding_transformation);
\draw [arrow] (embedding_transformation) -- (edge_embeddings);
\draw [arrow] (edge_embeddings) -- (explainer);
\draw [arrow] (explainer) -- (logits);
\draw [arrow] (logits) -- (trick);
\draw [arrow] (trick) -- (weights);

%\draw [arrow] (input.west) --  ++(-2cm,0) -- ($ (input) + (-2cm,0) $ |- sampled_graph) -- (sampled graph.west);
\draw [arrow] (weights) -- (sampled_graph);

\draw[arrow] 
  (input) -- (left_of_input) 
       -- (drop) 
       -- (weight_sample_mid);

\draw[arrow] 
  (left_of_input) -- (right_of_input) 
       -- (original_target);

\draw [arrow] (sampled_graph) -- (sample_target);
\draw [arrow] (sample_target) -- (sample_prediction);

\draw [arrow] (original_target) -- (original_prediction);

\draw [arrow] (sampled_graph) -- (right_of_sampled_graph) -- (topK);

\draw [arrow] (topK) -- (explanation);

\draw [<->, dashed] (sample_prediction) -- node[midway, above, font=\scriptsize] {$\min_\Omega H(Y_o,\hat{Y}_s)$} (original_prediction);

\end{tikzpicture}
\caption{TODO: The complete pipeline of PGExplainer.}
\label{fig:PGExplainer_pipeline}
\end{figure}

\begin{figure}
\centering
\begin{tikzpicture}
    
    \node (node_emb_i1) [data] {$z_i$};
    \node (node_emb_j1) [data, right=0.5cm of node_emb_i1] {$z_j$};
    
    \coordinate (i_j_mid) at ($ (node_emb_i1)!0.5!(node_emb_j1) $);
    
    \node (concat) at ($(i_j_mid) + (0,-1)$) {\Large $\oplus$};
    
    \node (edge_emb_ij) [data] at ($(i_j_mid) + (0,-2)$) {Edge Embedding of $(i,j)$};
    
    
    \node (node_emb_i2) [data, right=3cm of node_emb_j1] {$z_i$};
    \node (node_emb_j2) [data, right=0.5cm of node_emb_i2] {$z_j$};
    \node (node_emb_v2) [data, right=0.5cm of node_emb_j2] {$z_v$};
    
    \coordinate (i_v_mid) at ($ (node_emb_i2)!0.5!(node_emb_v2) $);
    
    \node (concat2) at ($(i_v_mid) + (0,-1)$) {\Large $\oplus$};
    
    \node (edge_emb_ijv) [data] at ($(i_v_mid) + (0,-2)$) {Edge Embedding of $(i,j)$ with target node $v$};


    \node (z_i) [data, above=2cm of node_emb_j2] {$z_i$};

    \node (h_2) [data, above=1cm of z_i] {$h_{2,i}$};
    \node (h_1) [data, left=0.5cm of h_2] {$h_{1,i}$};
    \node (h_3) [data, right=0.5cm of h_2] {$h_{3,i}$};
 
    \node (O) [module, above=1cm of h_2] {$O$};
    \node (F_3) [module, above=0.2cm of O] {$F_3$};
    \node (F_2) [module, above=0.2cm of F_3] {$F_2$};
    \node (F_1) [module, above=0.2cm of F_2] {$F_1$};
    \node (H_2) [data, right=0.5cm of F_2] {$H_2 = [h_{2,i}, h_{2,j},...]$};
    \node (H_1) [data, right=0.5cm of F_1] {$H_1 = [h_{1,i}, h_{1,j},...]$};
    \node (H_3) [data, right=0.5cm of F_3] {$H_3 = [h_{3,i}, h_{3,j},...]$};

    \node (input) [data, above=1cm of F_1] {$G_o$};

    \node (concat3) at ($ (h_2 |- z_i) + (0,1)$) {\Large $\oplus$};

    \begin{pgfonlayer}{background}
        \node[draw=gray, thick, rounded corners, fit=(F_1) (O) (H_1), fill=gray!30, inner sep=20pt, label=above:\text{Target GNN}] {};
    \end{pgfonlayer}



    \begin{pgfonlayer}{background}
        \node[draw=purple, dashed, rounded corners, fit=(node_emb_i1) (node_emb_j1) (node_emb_i2)(node_emb_v2) (edge_emb_ijv), inner sep=20pt, label=above:\text{Edge Embedding Transformation}] {};
    \end{pgfonlayer}
    
    \begin{pgfonlayer}{background}
        \node[draw=gray, thick, rounded corners, fit=(node_emb_i1) (node_emb_j1) (edge_emb_ij), fill=red!30, label=above:Graph Task] {};
    \end{pgfonlayer}
    
    \begin{pgfonlayer}{background}
        \node[draw=gray, thick, rounded corners, fit=(node_emb_i2) (node_emb_v2) (edge_emb_ijv), fill=orange!30, label=above:Node Task] {};
    \end{pgfonlayer}
\end{tikzpicture}
\caption{TODO: Visualization of the edge embedding transformation used to create inputs for the explainer network. Depending on the downstream task used in the target model the created edge embedding differs slightly.}
\end{figure}

\subsection{Regularization Terms}
\label{sec:Regularization_Terms}
To enhance the preservation of desired properties of explanations the authors propose various regularization terms. These are added to the learning objective, depending on the specific downstream task at hand.\bigskip

\textbf{Size and entropy constraints}

Inspired by GNNExplainer \cite{ying2019gnnexplainer}, to obtain compact and precise explanations, a constraint on the size of the explanations is added in the form of $||\Omega||_1$, the $l_1$ norm on latent variables $\Omega$. Additionally, to encourage the discreteness of edge weights, element-wise entropy is added as a constraint:
\begin{equation}
    H_{\hat{G}_s} = -\frac{1}{|\varepsilon|}\sum_{(i,j)\in \varepsilon} (\hat{e}_{ij}\log \hat{e}_{ij} + (1-\hat{e}_{ij})\log(1-\hat{e}_{ij})),
\end{equation}
for one explanatory graph $\hat{G_s}$ with $\varepsilon$ edges. For the collective setting, this is added as a mean over all instances in $\mathcal{I}$. \bigskip

Note that the following two constraints are not used in the original experimental setup, but serve as inspiration for constraints introduced in our NeuroSAT application\ref{} and are therefore included. \bigskip

\textbf{Budget constraint}

The authors propose the modification of the size constraint to a budget constraint, for a predefined available budget $B$. Let $|\hat{G}_s| \leq B$, then the budget regularization is defined as:
\begin{equation}
    R_b = \text{ReLU}(\sum_{(i,j)\in \varepsilon}\hat{e}_{ij}-B).
\end{equation}
Note that $R_b = 0$ when the explanatory graph is smaller than the budget. When out of budget, the regularization is similar to that of the size constraint. \bigskip

\textbf{Connectivity constraint}

To enhance the effect of the explainer detecting a small, connected subgrap, motivated through real-life motifs being inherently connected, the authors suggest adding the cross-entropy of adjacent edges. Let $(i,j)$ and $(i,k)$ be two edges that both connect to the node $i$, then $(i,k)$ should rather be included in the explanatory graph if the edge $(i,j)$ is selected to be included. This is formally defined as:
\begin{equation}
    H(\hat{e}_{ij},\hat{e}_{ik}) = -[1-\hat{e}_{ij}\log(1-\hat{e}_{ik})+\hat{e}_{ij}\log \hat{e}_{ik}].
\end{equation}
We note that in practice this is implemented only for the two highest edge weights for each edge. The definition therefore would change to $(i,j)$ and $(i,k)$ being the edges carrying the top two edge weights from the nodes connecting to node $i$.


\section{Extension to application on NeuroSAT}
\label{sec:NeuroSAT_extension}
In this section we propose additional restraints to fit the explanations of PGExplainer to the structure of SAT formulae. We start by giving a short introduction to NeuroSAT\cite{selsam2018learning} and how it may function as a downstream model.

 \bigskip
Proposed by Selsam et al. \cite{selsam2018learning}, NeuroSAT utilizes a MPNN, generally utilized to express GNNS, to solve SAT formulae. It is able to generalize in the sense that it may solve substantially larger and more difficult formulae than seen during training by running for more iterations. Though it may be used to calculate variable assignments that satisfy a formula, it is unable to provide proofs for formulae that are unsatisfiable.

In a separate experiment the authors were able to make NeuroSAT identify specific unsatisfiable cores, however this is assumed to be due to the network memorizing the subgraphs rather than learning a generic procedure that proves unsatisfiability. To verify this we aim to generate explanations for unsatisfiable formulae processed by a trained NeuroSAT model.

%Let $n$ denote the number of variables in a SAT formula and $m$ the number of clauses.
Since the following restraints are tailored to the application on bipartite graphs representing SAT formulae, we have to define the specifics of the input graph. A formula is encoded as an undirected bipartite graph $G_b$, with bipartition $(L,C)$. It contains one node for every literal $l \in L$, one for every clause $c \in C$ and edges for all combinations $(l,c)$ where $l$ appears in clause $c$. Additionally, connections exist between each literal and its negation, since messages are also passed along these (TODO: RELEVANT?). Note that these edges are not present in the biadjacency matrix $\mathbf{B}\in (0,1)^{L\times C}$. $\mathbf{B}$ is used as input for the NeuroSAT model, without explicit node features. In the following definitions we let $G_b$ be the original input graph for the PGExplainer, completely defined by its biadjacency matrix $\mathbf{B}$.

%aggregation: sum the outgoing messages of each of a node’s neighbors to form the incoming message

Since PGExplainer generates edge wise explanations, and we want to evaluate the SAT problem evaluations with unsatisfiable cores as ground truth, we need to adapt the framework to account for the definition of unsatisfiable cores. Since a core is a subset of clauses, predicting singular edges that represent a literal being present in a clause, may not provide sufficient results in the sense of human understandable explanations. Therefore, we propose a soft and a hard restraint that encourage the explainer to predict sets of edges that connect to the same node $c$, approximating the prediction of a complete clause.

The remainder of the explainer pipeline stays identical. The downstream task - NeuroSAT - calculates hidden node representations $\mathbf{h}^t$ for the input graph $G_b$, now modeling a SAT instance, at each iteration $t$. The representations in the last iteration $T$ are extracted as node embeddings $\mathbf{z}$ for clause and literal nodes respectively, and transformed into edge embeddings (see Equation \ref{eq:mlp_graph_input}). Though this is only done for node level tasks in the original, we also consider using a concatenation of multiple hidden representations $\mathbf{h}^{\frac{1}{2}T} \oplus \mathbf{h}^{\frac{3}{4}T} \oplus \mathbf{h}^T$ as node embeddings $\mathbf{z}$. These serve as the input of the explainer MLP and are processed as usual, with either of the following additional limitations. \bigskip


\textbf{Soft modified connectivity constraint}

To account for the definition of unsatisfiable cores - a subset of clauses in the original formula whose conjunction is still unsatisfiable - we add a constraint that reinforces the prediction of complete clauses. Therefore, if the explainer assigns a high score to an edge $(l_1,c)$, all edges $(l_k,c) \in E(c)$ that also connect to the clause node $c$ should receive a high score. Therefore, we introduce a soft constraint that punishes varying edge weights for the same clause. For our sampled bipartite Graph $\hat{G}_s$ with node sets $L$ and $C$ containing literal nodes and clause nodes respectively, we define:
\begin{equation*}
    R_C = \sum_{c \in C}  \text{Var}(\hat{E}_c) = \sum_{c \in C} \frac{1}{|E(c)|} \sum_{(l,c) \in E(c)} (\hat{e}_{l,c} - \bar{E_c})^2,
\end{equation*}
where $\hat{E}_c = \{\hat{e}_{l,c} \mid (l,c)\in E(\hat{G}_s)\}$ is the set of edge weights corresponding to edges incident to $c$ and $\bar{E_c} = \frac{1}{|\hat{E}_c|}\sum_{\hat{e}_{l,c} \in \hat{E}_c} \hat{e}_{l,c}$ denotes the mean of $\hat{E}_c$. TODO: $\mu_{\hat{e}_c}$ This is added to our objective function during training. \bigskip


\textbf{Hard constraint}

Since the soft constraint only encourages the prediction of entire clauses but does not enforce it, we also propose a hard constraint that modifies the predicion process. We restrain the edge logits $\omega_{i,j}$ calculated by the MLP to be identical for all edges that connect to the same clause. %Let $\Omega_c = \{\hat{\omega}_{i,c} \mid (i,c)\in E\}$ denote the set of logits corresponding to the edges connected to node $c$. We update these logits with:
%\begin{equation}
%    \mu_c = \frac{1}{|\Omega_c|} \sum_{\hat{\omega}_{i,c} \in \Omega_c} \hat{\omega}_{i,c}
%\end{equation}
For all clause nodes $c \in C$, we calculate the mean logit $\mu_c$ (TODO: OR $ \mu_{\omega_c}$) of all edges incident to $c$ with
\begin{equation}
    \mu_c = \frac{1}{|E(c)|} \sum_{(l,c)\in E(c)}\omega_{l,c}.
\end{equation}
The update rule is then defined as 
%\begin{equation}
%    \omega_{l,c}' =\mu_c, \qquad \forall(l,c) \in E(c) \mid 
%\end{equation}
\begin{equation}
    \omega_{l,c}'\leftarrow \mu_c,
\end{equation}
since edges in the biadjacency matrix are from literals to clauses at all times.


The reparameterization trick is still applied for each edge, but $\epsilon_c$ is sampled per clause instead of per edge, so that all edges that connect to a clause are forced to not only bear the same logit, but also the same importance score during training. The Equation \ref{eq:reparam_trick} thus changes to:
\begin{equation}
    \epsilon_c \sim \text{Uniform}(0,1), \qquad \hat{e}_{l,c}=\sigma((\log \epsilon_c - \log(1-\epsilon_c)+\omega_{l,c}/\tau).
\end{equation}


\section{Implementation details}
In the following, we provide the implementation details needed to reproduce our results. This includes the general replication of the PGExplainer \cite{} with the adapted downstream models in Section \ref{sec:Replication_of_PGExplainer} and the specifics for the application on NeuroSAT \cite{} in Section \ref{sec:Application_to_NeuroSAT}.


\subsection{Replication of PGExplainer}
\label{sec:Replication_of_PGExplainer}

For our replication we try to implement the methods and details as close to the original paper as possible. Thus, we follow the general pseudocode algorithms presented by the authors (TODO: see appendix \ref{}). Since the paper differs from the original codebase and is imprecise about certain descriptions (see Holdijk et al. \cite{holdijk2021re}), we aim to give a thorough description. This includes the tools we used, resulting changes regarding the data processing, the general architecture and hyperparameters of the downstream models, the architecture and hyperparameters of the explainer model, as well as concrete methods implemented in the model. \bigskip

\textbf{Libraries}: To reimplement the framework, we utilize a couple of libraries that we introduce shortly. Most notably, we use PyTorch Geometric \cite{Fey/Lenssen/2019}, a library built upon PyTorch \cite{paszke2019pytorch}, that provides methods to create and train GNNs. For evaluation of the explainer model, specifically for calculating the ROC-AUC score, we use TorchEval, a model evaluation library that is part of PyTorch. Furthermore, we integrate WandB \cite{wandb} to monitor model performance and allow for easy hyperparameter searches. To visualize the graphs and their explanations we employ NetworkX \cite{SciPyProceedings_11}.\bigskip

\textbf{Preprocessing}: We use the original datasets that are provided in the PGExplainer codebase. We transform the data to fit our PyTorch Geometric framework. Each graph is stored as a torch-geometric Data object. This holds the $d$-dimensional node features as tensors, the graph label index in the form of a long or all node labels in the form of a tensor of class indices, a ground truth edge mask that contains the edges present in the motif, as well as training-, evaluation- and test-node-masks for training the downstream model. In PyTorch Geometric edges are stored in the edge-index format as a COO tensor - a PyTorch coordinate format that stores tuples of element indices and their corresponding values. In the context of graph edges in PyTorch Geometric, for an edge $(i,j)$ the element index is the starting node $i$ and the corresponding value its incident node $j$. This is computed from the adjacency matrix $A$ as follows:
\begin{verbatim}
    edge_index = A.nonzero().t().contiguous()
\end{verbatim}
First, the matrix indices or coordinates of the edges - non-zero elements - are extracted. These are then transposed and lastly stored in contiguous memory. The resulting shape of the edge-index is $\mathbb{N}^{2\times \text{number-edges}}$ TODO:$[2, \text{number-edges}]$. Therefore, we only transform the data without changing its content. \bigskip

\textbf{Reproducibility}: Inspired by Holdijk et al. \cite{holdijk2021re} we implement the ability to seed the experiments performed on PGExplainer. PyTorch Geometric provides a way of seeding all modules that generate random numbers during the training process, including torch and python random.
To further increase reproducibility, we utilize PyTorch's \verb|use_deterministic_algorithms|, forcing the learning algorithm to only use deterministic algorithms. For the dataset splits we use a separate fixed seed that consistently creates the same sets across all training runs and experiments.\bigskip

\textbf{Downstream model specifics}: In PGExplainer two slightly different architectures of GNNs for node classification and graph classification are introduced. We recreate these in PyTorch Geometric, while changing the exact layers used in the network to test whether the claim that the explainer does apply to any downstream GNN model holds. These models implement the same downstream classification tasks on the given datasets that achieve accuracies of at least $85\%$, a baseline set in GNNExplainer \cite{}. The datasets as well as the exact accuracies of each of the models are presented in the experimental setup (see \ref{}). 

The model for both node- and graph classification consist of $3$ "GraphConv" layers, a Pytorch Geometric implementation of the Higher-order GNN by Morris et al \cite{morris2019weisfeiler} (see Equation \ref{eq:higher-order-gnn}). Since this layer allows for the passing of edge weights, weights of one are passed by default - e.g. during downstream model training - to simply maintain the adjacency matrix. Each of these layers has $20$ hidden units and is followed by a ReLU activation function. The first layer processes the $d$-dimensional input node features, while the remaining layers retain the hidden dimensionality of $20$. Holdijk et al. \cite{holdijk2021re} found that the original code wrongfully uses undocumented batch normalization layers in training mode during evaluation, which leads to a deviation in results, and thus completely omit the use of batch normalization. We choose to add batch normalization between the first two layers and their activation functions for both models, as done in the orginal codebase, without the training mode error. Additionally, we add an optional dropout layer after each activation function to improve the generalizability on more difficult tasks.

As mentioned in Section \ref{sec:Global_Explanations}, the original codebase uses a concatenation of the hidden states of each layer in the node classification model, rather than solely the last one. Thus, the node embeddings used in the explainer as well as in the downstream classification task each have a shape of $\mathbb{R}^{3(20)}$. For the classification a final linear layer is added to the model that maps each $60$-dimensional node embedding to $C$ classes. A softmax is applied to the model output to get class probabilities for each node. We also adopt this in our implementation. 

The model used for graph tasks differs slightly, in the sense that only the hidden embeddings of the last "GraphConv" layer are treated as node embeddings. In addition, before the final linear layer used for classification, both a max pooling and mean pooling operation are performed on the embeddings of each graph and the results concatenated to get a representation of a complete graph. Note that the paper only states to use max pooling, while in practice a concatenation of the max pooling and sum pooling is used. We adopt the combination of mean- and max pooling from the replication study \cite{holdijk2021re}, as this has been used in recent graph neural networks \cite{simonovsky2017dynamic} \cite{ma2021unsupervised} \cite{zhao2023faithful}. The max pooling operation extracts the maximum value of each feature dimension across the nodes of a graph, while the mean pooling extracts the mean value of each feature dimension across all nodes of a graph. Since the results of both pooling operations are in $\mathbb{R}^{20}$, the resulting $40$-dimensional graph embedding is fed into the linear layer, again mapping to $C$ classes, and a softmax is applied to get probabilities of a graph belonging to each class.\bigskip

TODO: LEAVE THESE OUT??
Max pool:
\begin{equation}
    \mathbf{r}_i = \text{max}_{n=1}^{\mathcal{N}_i} \mathbf{x}_n,
\end{equation}
with $\mathbf{x}$ denoting a vector of the feature matrix $\mathbf{X}\in \mathbb{R}^{\mathcal{N}_i\times F}$ of graph $i (\mathcal{G}_i)$.

Mean pool:
\begin{equation}
    \mathbf{r}_i = \frac{1}{\mathcal{N}_i}\sum_{n=1}^{\mathcal{N}_i} \mathbf{x}_n,
\end{equation}
with $\mathbf{x}$ denoting a vector of $\mathcal{N}_B$ 

TODO: WHERE?!?!?
The softmax function is defined as follows \cite{Goodfellow-et-al-2016}:
Given a vector \( \mathbf{z} = [z_1, z_2, \dots, z_K] \in \mathbb{R}^K \), the softmax function maps \( \mathbf{z} \) to a probability distribution over \( K \) classes:

\begin{equation}
    \text{Softmax}(\mathbf{z})_i = \frac{e^{z_i}}{\sum_{j=1}^{K} e^{z_j}} \quad \text{for } i = 1, 2, \dots, K,
\end{equation}
where $e$ denotes the exponential function.

Furthermore, following the original paper, all layer weights are initialized with Xavier initialization \cite{glorot2010understanding}, while biases are initialized with $0$. The datasets are split into training, testing and validation sets with an 80/10/10 split ratio. The models are trained for 1000 epochs, with a learning rate of $1.0 \times 10^{-3}$ and Adam \cite{kingma2014adam} is used as an optimizer.

Note that the main differences between our downstream model and the model described in PGExplainer \cite{luo2020parameterized} lie in the used graph layer, as well as the addition of dropout, two batch normalizations before the activations and the usage of a slightly different global pooling.

We want to stress that though not documented in the paper, early stopping is utilized in the downstream model training and the model state with the highest validation accuracy is selected. For models that achieve the highest accuracy at multiple epochs, the model state with the lowest validation loss is chosen. If the validation loss does not decrease below the minimum for 500 epochs, the training is stopped early. This is very important for a fair comparison of the models, as we found that the explanations of an overfit downstream model vary from the "best" one. (TODO: EXPERIMENT PROOF NEEDED!?) \bigskip

In a more recent version of the PGExplainer \cite{10423141}, the authors add a formal description of the downstream model layer used in their framework. The used GNN layer is defined as:
\begin{equation}
    f(H(l),A)=\sigma(W(l)AH(l)),
\end{equation}
where $H(l)$ are the hidden representations of nodes in the $l$-th layer, $A$ is the normalized Laplacian matrix, and $W(l)$ is the weight matrix.

Note that a PyTorch Geometric implementation of the GCN layer described in equation \ref{eq:GCN} is used in the replication paper \cite{holdijk2021re} with a ReLU activation function. Thus, the sole difference between the two is the order of the matrix multiplication. \bigskip

We use a PyTorch Geometric implementation of the Higher-order GNN layer with a ReLU activation function, as defined in Equation \ref{eq:higher-order-gnn}. On the one hand, it works on bipartite graphs by definition. On the other hand, it explicitly allows for edge weights to be passed, which is essential for "applying" the sampled, approximate discrete explanation mask to the input graph during training. This is necessary since in practice, applying a hard mask to the input graph would prevent the computation of gradients during back propagation. Thus, instead of actually removing edges of the graph, the edge importance scores that are learned by the MLP are treated as edge weights in the prediction process of the downstream model, with higher scores indicating edges being more relevant for the prediction. During training, the prediction for the original input graph is computed from the node features and edge-index alone, while for the sampled graph the edge weights are passed additionally. (TODO: Visualization?)\bigskip

\textbf{Explainer Architecture}:
We implement the explainer MLP architecture described in the paper. It consists of two linear layers, with a ReLU activation function applied after the first. The first layer maps the input edge embedding dimension to 64 hidden units, and the second maps the hidden units to a single output scalar. The input edge embeddings for the MLP are calculated as described in Section \ref{sec:Global_Explanations}. For the described downstream models, the resulting MLP input - a concatenation of the node embeddings - has a shape of $\mathbb{R}^{2\cdot20}$ for graph tasks and $\mathbb{R}^{3\cdot60}$ for node tasks. Again, this is taken from the original codebase, as the paper does not consider the node embeddings as a concatenation of all GNN layers for node tasks.

We implement the calculation of edge embeddings as follows:
\begin{verbatim}
    def getEdgeEmbeddings(self, modelGNN, x, edge_index, nodeToPred=None):
        emb = modelGNN.getNodeEmbeddings(x, edge_index)
        i, j = edge_index[0], edge_index[1]
        
        if nodeToPred is not None:
            node_emb = emb[nodeToPred].repeat(len(i), 1)
            embCat = torch.cat([emb[i], emb[j], node_emb], dim=1)
        else:
            embCat = torch.cat([emb[i], emb[j]], dim=1)
        return embCat
\end{verbatim}
The trained, fixed downstream model \verb|modelGNN| is passed to the explainer and returns its node embeddings for the input graph, defined by its node features \verb|x| and the \verb|edge_index|. Depending on the task at hand, the node embeddings of each two connected nodes, as well as the node embeddings of the node to be predicted \verb|nodeToPred| in the case of a node task, are concatenated and returned as edge embeddings. \bigskip

We adopt the initializations from the downstream models and initialize all layer weights with Xavier \cite{glorot2010understanding} and biases with zeroes. During training, the Adam optimizer \cite{kingma2014adam} is used to update the model parameters based on gradients. Additionally, gradients are clipped during training using PyTorch's \verb|clip_grad_norm_| with a maximum norm of 2. The authors define the temperature used in the reparameterization trick (see Equation \ref{eq:reparam_trick}) with an annealing temperature schedule, as proposed by Abid et al. \cite{abid2019concrete}:
\begin{equation}
    \tau^{(t)} = \tau_0(\frac{\tau_T}{\tau_0})^{\frac{t}{T}},
\end{equation} 
where $t$ is the current epoch and $T$ is the total number of epochs. $\tau_0$ and $\tau_T$ are hyperparameters that define the initial and final temperature, respectively.
The reason for this is that small temperatures tend to generate more discrete graphs, as they more closely approximate samples from the Gumbel-Softmax distribution. While this is desirable at convergence, it may hinder the optimization early in training due to a reduced gradient signal. We note that in the original codebase $t$ is initialized with $0$, leading to a temperature $\tau^{(0)} = \tau_0$ in the first epoch, and $\tau^{(T-1)} = \tau_0(\frac{\tau_T}{\tau_0})^{\frac{T-1}{T}}$ in the last epoch. We initialize $t$ with $1$ to get a temperature of $\tau^{(T)} = \tau_T$ in the last epoch, as we believe this to be more inline with the proposition by Abid et al. \cite{abid2019concrete}.

In the PGExplainer paper, Luo et al. \cite{} propose the use of high temperatures, with $\tau_0 = 5.0$ and $\tau_T=2.0$. However, Hodlijk et al. \cite{} found that all hyperparameters described in the paper are overly simplified and in practice, different parameters are used for each dataset. Therefore, we conduct hyperparameter searches for each task in \ref{}. We add that the number of sampled graphs $K$ is not defined in the original work, and the codebase suggests the use of $K=1$. We implement the ability to sample multiple graphs as described in the pseudocode \ref{} and also consider this in our hyperparameter searches. \bigskip

\textbf{Extension description}:
This is not described in the original paper, but since undirected graphs contain bidirectional edges, effectively each edge carries two importance scores. The authors alleviate this in the code by symmetrizing the edge weight matrix, corresponding to the adjacency matrix of the graph. We propose meaning the logits of each pair of bidirectional edges after the MLP output has been calculated. We implement the logic of meaning each edge's logits as follows: 
\begin{verbatim}
    edge_pairs = edge_index.t()
    # Sort node pairs so that (i, j) and (j, i) are treated the same
    canonical_pairs, _ = edge_pairs.sort(dim=1)
    # Find unique undirected edges and get mapping indices
    unique_pairs, inverse_indices = torch.unique(canonical_pairs, dim=0)
    
    # Average weights for duplicate edges
    for i in range(unique_pairs.size(0)):
        mask = inverse_indices == i
        mean_weight = torch.mean(w_ij[mask])
        w_ij[mask] = mean_weight
\end{verbatim}
All edges pairs from the edge-index that connect the same two nodes, e.g. $(i,j)$ and $(j,i)$, are treated as a unique pair. For each of these unique pairs the mean of the logits $\omega_{i,j}$ and $\omega_{j,i}$ corresponding to its two edges is calculated and used to update each of the two logits. To account for this in the reparameterization trick, the unique pairs are further passed there to sample $\epsilon$ for each edge pair, rather than each edge. This process guarantees edge pairs that connect the same two nodes to always carry identical importance scores. \bigskip

\textbf{TODO: Explainer implementation}:
TODO: WHERE? We already established in Section \ref{sec:Global_Explanations} that the node prediction of an $L$-layer GNN is completely determined by its local computation graph, defined by its $L$-hop neighborhood \cite{ying2019gnnexplainer}. This is capitalized on during explainer training by computing the neighborhood graphs of each input node and treating this subgraph as a base for the node predictions, rather than using the full graph (TODO: SEE ALGORITHM). Since the downstream models for node tasks consist of three graph layers, the $3$-hop subgraph of each node is computed. We implement this using the \verb|k_hop_subgraph()| function from PyTorch Geometric. \bigskip

TODO: WHERE? We compute the loss as described in Equation \ref{eq:mlp_loss}, with each regularization term being added according to a regularization coefficient hyperparameter. It is noteworthy that the loss function used in the original codebase can formally be defined as:
\begin{equation}
    \min_\Psi -\frac{1}{K}\sum_{i\in \mathcal{I}}\sum_{k=1}^K -\log P_\Phi(\hat{Y}_s = c_i|G=\hat{G}_s^{(i,k)}),
\end{equation}
where $c_i$ denotes the class label according to the prediction on the original graph $G_o^{(i)}$ and $K=1$. However, this is not mentioned in the paper and therefore not adopted. 

TODO: DOES THIS BELONG HERE? PROBABLY GOOD IN GENERAL AS IT SPECIFIES TRAINING?
A further undocumented specification is that for downstream models that perform node tasks, only node instances that belong to the "motif classes" are selected for training and evaluation, since for nodes that do not belong to these classes there does not exist a specific ground truth motif that may serve as an explanation. This is only documented for one graph classification dataset, where only one of two possible classes has a dedicated motif.\bigskip

\textbf{Evaluation implementation}:
To quantitatively evaluate the explanations of each prediction (TODO: See exp. setup), the authors utilize the ROC-AUC as a metric to compare the predicted importance scores of the edges to the ground truth edges. Since the exact procedure is not further described by the authors, we extract it from the codebase as far as possible. For graph tasks the metric is computed globally, meaning that for all graph instances the edge predictions and ground truths are gathered, and the global thresholds are computed, while for node tasks the metric is computed locally for the $3$-hop subgraph of each node instance and a mean is calculated later on.
Since a reason for this difference in calculation is not provided, we choose to only consider the mean of the local values, regardless of task. TODO: We justify this with our application in the inductive setting, where global explanations are still provided, but we focus on generalizing to unseen instances.

TODO: Speculation? This may be motivated by the fact that motif nodes may be selected for explanation, where the computation graph only consist of motif nodes and consequently only ground truth edges. This may lead to an inflated score when calculated in a global way, as the ROC-AUC score is not computable for these cases and explicitly disregarded in the local computation.

We implement this calculation using the \verb|BinaryAUROC| from TorchEval, which operates identical to the \verb|roc_auc_score| from scikit-learn \cite{pedregosa2011scikit} in the binary case, used in PGExplainer.

The qualitatively explanations provided by the explainer are implemented as a NetworkX visualization of the graph instance in the case of graph tasks, or of the $3$-hop subgraph of the instance node in the case of node tasks. Only the edges with the top-$2k$ importance scores are drawn, where $k$ equals $\text{\#motif-edges}$, since edges are bidirectional and guaranteed to carry identical weights.

The inference time is measured as the time it takes to generate an explanation for any instance. This starts with the computation of node embeddings from the downstream model and ends with the reparameterization trick - the simple application of a Sigmoid to the edge logits during evaluation - that returns the edge importance scores. This process is illustrated in Figure \ref{fig:PGExplainer_pipeline}, starting from input graph $G_o$ and ending at the sampled graph $\hat{G}_s$. \bigskip

\textbf{TODO: Challenges}: NECESSARY? MENTION DIFFICULTIES IN EACH SUBSECTION? Difficulties getting the code to work, as we started working with the paper only. Found that hyperparameters had to be finetuned, exact downstream model is relevant (e.g. batch norm present or not, overfit models obviously generate bad explanations, need for early stopping). Different loss used in code than in paper; we adopt the loss described in the paper. No documentation of the motif node selection/training only performed on graphs/nodes that contain gt! Imprecise about dataset description (BA-2Motif has different features than the rest of the syn datasets). No description of the AUROC calculation, had to be extracted from code (global vs local) \bigskip

\subsection{Application on NeuroSAT}
\label{sec:Application_to_NeuroSAT}

In this section we describe the necessary changes for explaining predictions of NeuroSAT \cite{} with our PGExplainer framework. We want to stress that the NeuroSAT codebase was provided by a fellow student and only the changes mentioned in this section are part of our work. The source code for NeuroSAT is publicly available.\footnote{Daniel Selsam et al. “NeuroSAT” (2018). URL: \url{https://github.com/dselsam/neurosat}}

Since NeuroSAT can be regarded as a black box MPNN in the context of our work, we only describe how it passes messages superficially. In each iteration, each clause receives messages from its neighboring literals to update its embedding. Then, each literal receives messages not only from its neighboring clauses, but also from its complementary literal, to update its embedding. The number of iterations is set to $26$ for the model we are using, which was trained in the standard manner on both satisfiable and unsatisfiable SAT problems - formulae in CNF with the goal of determining whether they are satisfiable. The downstream task of NeuroSAT can hence be understood as a graph classification task.

Besides the prediction of satisfiability, NeuroSAT returns the node embeddings of both clauses and literals at each iteration. We use these to generate edge embeddings for all edges in the biadjacency matrix as input for the explainer MLP (see Section \ref{sec:NeuroSAT_extension}), using the same procedure as in the previous replication study (see Section \ref{sec:Replication_of_PGExplainer}). Since the SAT instances are defined as biadjacency matrices, edges only exist in one direction - from literals to clauses. Thus, we dismiss the meaning of edge logits and qualitatively evaluate using the top-$k$ edges, rather than the previous top-$2k$.

The only change we have to make is allowing the NeuroSAT forward pass to receive edge weights as a parameter. If no edge weights are passed, the forward pass behaves as usual and the biadjacency matrix contains discrete values of 0 or 1. However, when predicting sampled graphs from the explainer, we pass the sampled edge weights and multiply these with the biadjacency matrix, to receive a weighted biadjacency matrix. NeuroSAT then calculates the prediction for the weighted matrix. This change is implemented as follows:
\begin{verbatim}
    connections = torch.sparse_coo_tensor(
            indices=edges,
            values=torch.ones(problem.n_cells, device=self.device) 
                    if edge_weights==None else edge_weights,
            size=torch.Size([n_literals, n_clauses])
        ).to_dense()
\end{verbatim}
Usually, the \verb|connections| tensor - a biadjacency matrix - is initialized with ones at the coordinates of the edge-index tensor \verb|edges|, containing the \verb|problem.n_cells| edges of a SAT problem. However, when \verb|edge_weights| is passed into the function, the edge weights are used as initialization, rather than ones. This can be understood as a multiplication of the edge weights with the ones representing edges.

Furthermore, the input of the explainer changes to a SAT problem instance, containing only the edge-index representation of a SAT formula, as well as a label 1 or 0, denoting satisfiability or unsatisfiability, respectively. Node features are not used in NeuroSAT and therefore irrelevant for the explainer.

TODO: Implementation of constraints?

To evaluate the explanations provided by the PGExplainer, we require ground truths to calculate the ROC-AUC and understand the visualizations of explanations. Since our goal is testing whether the explanations of the NeuroSAT predictions align with human-understandable concepts, we propose the use of MUSs as ground truth. Therefore, we utilize the deletion-based MUS extractor \verb|MUSX| from PySAT \cite{imms-sat18} to generate a MUS for each unsatisfiable problem. This MUS is transformed to match the corresponding edges of its graph representation and treated as expected ground truth. This allows us to calculate the local ROC-AUC for each SAT problem as described in Section \ref{sec:Replication_of_PGExplainer}. We visualize the provided explanations and ground truths as a pyvis \cite{perrone2020network} \verb|Network| for qualitative evaluation. \bigskip

We adopt the MLP architecture described in Section \ref{sec:Replication_of_PGExplainer}, but also experiment with a more complex architecture (see Section \ref{}). Besides these changes, the explainer operates as previously explained.