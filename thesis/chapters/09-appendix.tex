\chapter{Supplementary Material}

\section{Material from original PGExplainer}
\label{sec:PGE_material}

\begin{algorithm}
    \caption{Training Algorithm for Explaining Node Classification from \cite{luo2020parameterized}.}
    \label{alg:node-alg}
    \begin{algorithmic}[1]
    \REQUIRE Input graph $G_o = (\mathcal{V}, \mathcal{E})$, node features $X$, node labels $Y$, set of instances to be explained $\mathcal{I}$, trained GNN model: $\text{GNNE}_{\Phi_0}(\cdot)$ and $\text{GNNC}_{\Phi_1}(\cdot)$, parameterized explainer MLP $\Psi$.
    \FOR{each node $i \in \mathcal{I}$}
        \STATE $G^{(i)}_o \leftarrow$ extract the computation graph for node $i$.
        \STATE $Z^{(i)} \leftarrow \text{GNNE}_{\Phi_0}(G^{(i)}_o, X)$.
        \STATE $Y^{(i)} \leftarrow \text{GNNC}_{\Phi_1}(Z^{(i)})$.
    \ENDFOR
    \FOR{each epoch}
        \FOR{each node $i \in \mathcal{I}$}
            \STATE $\Omega \leftarrow$ latent variables calculated with (10).
            \FOR{$k \leftarrow 1$ to $K$}
                \STATE $G^{(i,k)}_s \leftarrow$ sampled from (4).
                \STATE $\hat{Y}^{(i,k)}_s \leftarrow \text{GNNC}_{\Phi_1}(\text{GNNE}_{\Phi_0}(G^{(i,k)}_s, X))$.
            \ENDFOR
        \ENDFOR
        \STATE Compute loss with (9).
        \STATE Update parameters $\Psi$ with backpropagation.
    \ENDFOR
    \end{algorithmic}
    \end{algorithm}
    
    \vspace{0.5cm}
    
    \begin{algorithm}
    \caption{Training Algorithm for Explaining Graph Classification from \cite{luo2020parameterized}.}
    \label{alg:graph-alg}
    \begin{algorithmic}[1]
    \REQUIRE A set of input graphs with $i$-th graph represented by $G^{(i)}_o$, node features $X^{(i)}$, label $Y^{(i)}$, trained GNN model: $\text{GNNE}_{\Phi_0}(\cdot)$ and $\text{GNNC}_{\Phi_1}(\cdot)$, parameterized explainer MLP $\Psi$.
    \FOR{each graph $G^{(i)}_o$}
        \STATE $Z^{(i)} \leftarrow \text{GNNE}_{\Phi_0}(G^{(i)}_o, X^{(i)})$.
        \STATE $Y^{(i)} \leftarrow \text{GNNC}_{\Phi_1}(Z^{(i)})$.
    \ENDFOR
    \FOR{each epoch}
        \FOR{each graph $G^{(i)}_o$}
            \STATE $\Omega \leftarrow$ latent variables calculated with (11).
            \FOR{$k \leftarrow 1$ to $K$}
                \STATE $G^{(i,k)}_s \leftarrow$ sampled from (4).
                \STATE $\hat{Y}^{(i,k)}_s \leftarrow \text{GNNC}_{\Phi_1}(\text{GNNE}_{\Phi_0}(G^{(i,k)}_s, X^{(i)}))$.
            \ENDFOR
        \ENDFOR
        \STATE Compute loss with (9).
        \STATE Update parameters $\Psi$ with backpropagation.
    \ENDFOR
    \end{algorithmic}
\end{algorithm}

\begin{table}[h]
    \centering
    \scriptsize
    \begin{tabularx}{\linewidth}{l|X X X X|X X}
    \hline
    \textbf{Accuracy} & \textbf{BA-Shapes} & \textbf{BA-Community} & \textbf{Tree-Cycles} & \textbf{Tree-Grid} & \textbf{BA-2Motif} & \textbf{MUTAG} \\
    \hline
    \textbf{Training}   & 0.98 & 0.99 & 0.99 & 0.92 & 1.00 & 0.87 \\
    \textbf{Validation} & 1.00 & 0.88 & 1.00 & 0.94 & 1.00 & 0.89 \\
    \textbf{Testing}    & 0.97 & 0.93 & 0.99 & 0.94 & 1.00 & 0.87 \\
    \hline
    \end{tabularx}
    \caption[Accuracies of original GNN downstream task]{Compact accuracy table for Node and Graph Classification. Reprinted from \cite{luo2020parameterized}.}
    \label{tab:compact-accuracy}
\end{table}


\section{Data visualization}
\label{sec:data_vis}

\begin{figure}[H]
    \centering
    \begin{subfigure}[b]{0.4\textwidth}
        \includegraphics[width=\textwidth]{img/BA-Shapes-VIS-COMP-GRAPH.pdf}
        \caption{BA-Shapes}
    \end{subfigure}
    \hfill
    \begin{subfigure}[b]{0.4\textwidth}
        \includegraphics[width=\textwidth]{img/BA-Community-VIS-COMP-GRAPH.pdf}
        \caption{BA-Community}
    \end{subfigure}
    
    \vspace{0.5cm}
    
    \begin{subfigure}[b]{0.4\textwidth}
        \includegraphics[width=\textwidth]{img/Tree-Cycles-VIS-COMP-GRAPH.pdf}
        \caption{Tree-Cycles}
    \end{subfigure}
    \hfill
    \begin{subfigure}[b]{0.4\textwidth}
        \includegraphics[width=\textwidth]{img/Tree-Grid-VIS-COMP-GRAPH.pdf}
        \caption{Tree-Grid}
    \end{subfigure}
    
    \vspace{0.5cm}
    
    \begin{subfigure}[b]{0.4\textwidth}
        \includegraphics[width=\textwidth]{img/BA-2Motif-VIS-UNLABELED.pdf}
        \caption{BA-2Motif}
    \end{subfigure}
    \hfill
    \begin{subfigure}[b]{0.4\textwidth}
        \includegraphics[width=\textwidth]{img/MUTAG-VIS-LARGE-UNLABELED.pdf}
        \caption{MUTAG}
    \end{subfigure}

    \caption[Visualization of original PGExplainer datasets]{Visualization of all six datasets. For node datasets (a-d) the target prediction node where the computational graph is computed from is colored in light blue.}
\end{figure}

\section{Replication sweeps}
\label{sec:sweeps}
The following tables contain the hyperparameter search configurations for each downstream task. The first row includes the configuration used in the original codebase \cite{luo2020parameterized} and the second row contains the configurations used in \cite{holdijk2021re}. For BA-Community (see Table \ref{tab:BA-Community_sweep}) both configurations are identical, and the second row is thus omitted. The last section of each table contains the set of the values tested for each parameter. The optimal settings for our explainer implementation are highlighted in each column. Note that we optimize BA-2Motif (see Table \ref{tab:BA-2Motif_sweep}) towards a minimal metric score, as discussed in \ref{}.

\newcolumntype{Y}{>{\centering\arraybackslash}X}
\begin{table}[h]
  \centering
  \scriptsize
  \begin{tabular}{|c|c|c|c|c|c|c|c|c|c|}
  \hline
  \multicolumn{10}{|c|}{\textbf{BA-Shapes}} \\ \hline
  $a$ & $K$ & $b$ & $E$ & $\eta$ & $S$ & $\alpha_e$ & $\alpha_s$ & $\tau_0$ & $\tau_T$ \\ \hline
  $N$ & 1 & 0.0 & 10 & 0.003 & - & 1.0 & 0.05 & 1.0 & 0.05 \\ \hline
  $N$ & 1 & 0.0 & 10 & 0.003 & - & 1.0 & 0.05 & 5.0 & 2.0 \\ \hline
  5 & \textbf{1} & 0.0 & 10 & 0.0003 & 74 & \textbf{0.1} & 0.005 & 5.0 & \textbf{1} \\
   & 5 &  &  & \textbf{0.003} & 75 & 0.5 & \textbf{0.05} &  & 2 \\
   & 10 &  &  & 0.03 & 76 & 1.0 & 0.1 &  & 5 \\ \hline
  \end{tabular}
  \caption[BA-Shapes Sweep]{First two rows show original and replication configurations; following rows show parameter sweep results. Bolded values indicate best performance.}
\end{table}

\begin{table}[h]
  \centering
  \scriptsize
  \begin{tabular}{|c|c|c|c|c|c|c|c|c|c|}
  \hline
  \multicolumn{10}{|c|}{\textbf{BA-Community}} \\ \hline
  $a$ & $K$ & $b$ & $E$ & $\eta$ & $S$ & $\alpha_e$ & $\alpha_s$ & $\tau_0$ & $\tau_T$ \\ \hline
  $\text{All}$ & 1 & 0.5 & 20 & 0.003 & - & 1.0 & 0.05 & 1.0 & 1.0 \\ \hline
  64 & 1 & \textbf{0.0} & 20 & \textbf{0.003} & 74 & \textbf{1.0} & 0.05 & 1.0 & 1.0 \\
   & \textbf{5} & 0.5 &  & 0.0003 & 75 & 0.1 & \textbf{0.1} &  & \textbf{5.0} \\
   & 10 &  &  &  & 76 &  &  &  &  \\ \hline
  \end{tabular}
  \caption[BA-Community Sweep]{First row shows original configuration; following rows show parameter sweep results. Bolded values indicate best-performing settings.}
  \label{tab:BA-Community_sweep}
\end{table}

\begin{table}[h]
    \centering
    \scriptsize
    \begin{tabularx}{\linewidth}{|c|c|c|Y|Y|c|c|c|c|Y|}
    \hline
    \multicolumn{10}{|c|}{\textbf{Tree-Cycles}} \\ \hline
    $\alpha_e$ & $E$ & $\eta$ & $K$ & $b$ & $S$ & $\alpha_s$ & $\tau_T$ & $\tau_0$ & $a$ \\ \hline
    0.01 & 20 & 0.003 & 1 & 0.0 & - & 0.0001 & 5.0 & 5.0 & \text{All} \\ \hline
    10.0 & 20 & 0.003 & 1 & 0.0 & - & 0.1 & 5.0 & 1.0 & \text{All} \\ \midrule
    0.01 & 20 & \textbf{0.0003} & 1 & 0.0 & 74 & \textbf{0.0001} & \textbf{1.0} & 1.0 & 5 \\ 
    \textbf{1.0} &  & 0.003 & \textbf{5} &  & 75 & 0.05 & 5.0 &  &  \\ 
    10.0 &  &  & 10 &  & 76 & 0.1 &  &  &  \\ \hline
    \end{tabularx}
    \caption[Tree-Cycles Sweep]{TODO: CONSIDER OPTIMUM TO BE CLASSIFICATION EITHER REALLY HIGH OR REALLY LOW?? First row contains the values used in the original code; second row for replication. Highlighted values are the ones that achieved scores closest to $0$ or $1$, depending on the seed and the "direction" it is learning.}
\end{table}

\begin{table}[h]
    \centering
    \scriptsize
    \begin{tabularx}{\linewidth}{|c|c|c|Y|Y|c|c|c|c|Y|}
    \hline
    \multicolumn{10}{|c|}{\textbf{Tree-Grid}} \\ \hline
    $\alpha_e$ & $E$ & $\eta$ & $K$ & $b$ & $S$ & $\alpha_s$ & $\tau_T$ & $\tau_0$ & $a$ \\ \hline
    1.0 & 30 & 0.01 & 1 & 0.0 & - & 0.01 & 5.0 & 5.0 & \text{All} \\ \hline
    1.0 & 30 & 0.003 & 1 & 0.0 & - & 1.0 & 2.0 & 5.0 & \text{All} \\ \midrule
    0.1 & 30 & 0.0003 & 1 & 0.0 & 74 & 0.01 & \textbf{2.0} & 5.0 & 24 \\ 
    \textbf{1.0} &  & \textbf{0.003} & \textbf{5} &  & 75 & \textbf{0.5} & 5.0 &  &  \\ 
    10 &  & 0.01 & 10 &  & 76 & 1.0 &  &  &  \\
     &  & 0.05 &  &  &  &  &  &  &  \\ \hline
    \end{tabularx}
    \caption[Tree-Grid Sweep]{First row contains the values used in the original code; second row for replication. Highlighted values are the best performing. size-reg of $0.01$ leads to all edges being one. With original experimental setup, mostly complete randomness}
\end{table}

\begin{table}[h]
    \centering
    \scriptsize
    \begin{tabularx}{\linewidth}{|c|c|c|Y|Y|c|c|c|c|Y|}
    \hline
    \multicolumn{10}{|c|}{\textbf{BA-2Motif}} \\ \hline
    $\alpha_e$ & $E$ & $\eta$ & $K$ & $b$ & $S$ & $\alpha_s$ & $\tau_T$ & $\tau_0$ & $a$ \\ \hline
    0.0 & 10 & 0.003 & 1 & 0.0 & - & 0.00 & 0.0 & 1.0 & \text{All} \\ \hline
    0.01 & 20 & 0.005 & 1 & 0.0 & - & 0.03 & 1.0 & 5.0 & \text{All} \\ \midrule
    0.01 & 10 & 0.0003 & 1 & 0.0 & 74 & 0.03 & \textbf{1.0} & 5.0 & 30 \\ 
    \textbf{0.1} & \textbf{20} & 0.003 & 5 &  & 75 & & 5.0 &  &  \\ 
    &  & 0.005 & \textbf{10} &  & 76 &  &  &  &  \\
     &  & \textbf{0.01} &  &  &  &  &  &  &  \\ \hline
    \end{tabularx}
    \caption[BA-2Motif Sweep]{First row contains the values used in the original code; second row for replication. Highlighted values are the one that achieve the lowest individual AUROCs, as the explainer seems to learn the opposite for BA-2Motif.}
    \label{tab:BA-2Motif_sweep}
\end{table}

\begin{table}[h]
    \centering
    \scriptsize
    \begin{tabularx}{\linewidth}{|c|c|c|Y|Y|c|c|c|c|Y|}
    \hline
    \multicolumn{10}{|c|}{\textbf{MUTAG}} \\ \hline
    $\alpha_e$ & $E$ & $\eta$ & $K$ & $b$ & $S$ & $\alpha_s$ & $\tau_T$ & $\tau_0$ & $a$ \\ \hline
    1.0 & 10 & 0.01 & 1 & 0.0 & - & 0.01 & 5.0 & 5.0 & \text{All} \\ \hline
    1.0 & 30 & 0.0003 & 1 & 0.0 & - & 0.005 & 5.0 & 5.0 & \text{All} \\ \midrule
    0.1 & 10 & 0.0003 & 1 & 0.0 & 74 & 0.01 & \textbf{1.0} & 5.0 & 30 \\ 
    \textbf{1.0} & \textbf{20} & 0.003 & 5 &  & 75 & \textbf{0.005} & 5.0 &  & \\ 
     & 30 & \textbf{0.01} & \textbf{10} &  & 76 &  &  &  &  \\ \hline
    \end{tabularx}
    \caption[MUTAG Sweep]{First row contains the values used in the original code; second row for replication. Highlighted values are the best performing.}
\end{table}

\clearpage
\section{Multiple Explanation Visualizations}
The following figures contain 16 randomly sampled explanations for the best explainer model of every dataset. For graph tasks all nodes of the input graph are included, while for node tasks only the nodes of the local computation graph from the prediction motif node are shown.

% REMOVE draw=black TO REMOVE EDGES
\begin{figure}[htbp]
    \centering
    \begin{tikzpicture}
      % 4x4 grid: row = 0 to 3, col = 0 to 3
      \foreach \row in {0,...,3} {
        \foreach \col in {0,...,3} {
          \pgfmathtruncatemacro{\num}{\col + 4 * \row}
          \node[draw=black, thin, inner sep=0pt] at (\col*4, -\row*4)
            {\includegraphics[width=3.5cm]{img/BA-Shapes/graph_\num_explanation.pdf}};
        }
      }
    \end{tikzpicture}
    \caption{Grid of BA-Shapes explanations (top-6 edges)}
    \label{fig:grid-BA-Shapes-explanations}
\end{figure}

\begin{figure}[htbp]
    \centering
    \begin{tikzpicture}
      % 4x4 grid: row = 0 to 3, col = 0 to 3
      \foreach \row in {0,...,3} {
        \foreach \col in {0,...,3} {
          \pgfmathtruncatemacro{\num}{\col + 4 * \row}
          \node[draw=black, thin, inner sep=0pt] at (\col*4, -\row*4)
            {\includegraphics[width=3.5cm]{img/BA-Community/graph_\num_explanation.pdf}};
        }
      }
    \end{tikzpicture}
    \caption{Grid of BA-Community explanations (top-6 edges)}
    \label{fig:grid-BA-Community-explanations}
\end{figure}

\begin{figure}[htbp]
    \centering
    \begin{tikzpicture}
      % 4x4 grid: row = 0 to 3, col = 0 to 3
      \foreach \row in {0,...,3} {
        \foreach \col in {0,...,3} {
          \pgfmathtruncatemacro{\num}{\col + 4 * \row}
          \node[draw=black, thin, inner sep=0pt] at (\col*4, -\row*4)
            {\includegraphics[width=3.5cm]{img/Tree-Cycles/graph_\num_explanation.pdf}};
        }
      }
    \end{tikzpicture}
    \caption{Grid of Tree-Cycles explanations (top-6 edges)}
    \label{fig:grid-Tree-Cycles-explanations}
\end{figure}

\begin{figure}[htbp]
    \centering
    \begin{tikzpicture}
      % 4x4 grid: row = 0 to 3, col = 0 to 3
      \foreach \row in {0,...,3} {
        \foreach \col in {0,...,3} {
          \pgfmathtruncatemacro{\num}{\col + 4 * \row}
          \node[draw=black, thin, inner sep=0pt] at (\col*4, -\row*4)
            {\includegraphics[width=3.5cm]{img/Tree-Grid/graph_\num_explanation.pdf}};
        }
      }
    \end{tikzpicture}
    \caption{Grid of Tree-Grid explanations (top-12 edges)}
    \label{fig:grid-Tree-Grid-explanations}
\end{figure}

\begin{figure}[htbp]
    \centering
    \begin{tikzpicture}
      % 4x4 grid: row = 0 to 3, col = 0 to 3
      \foreach \row in {0,...,3} {
        \foreach \col in {0,...,3} {
          \pgfmathtruncatemacro{\num}{\col + 4 * \row}
          \node[draw=black, thin, inner sep=0pt] at (\col*4, -\row*4)
            {\includegraphics[width=3.5cm]{img/BA-2Motif/graph_\num_explanation.pdf}};
        }
      }
    \end{tikzpicture}
    \caption{Grid of BA-2Motif explanations (top-5 edges)}
    \label{fig:grid-BA-2Motif-explanations}
\end{figure}

\begin{figure}[htbp]
    \centering
    \begin{tikzpicture}
      % 4x4 grid: row = 0 to 3, col = 0 to 3
      \foreach \row in {0,...,3} {
        \foreach \col in {0,...,3} {
          \pgfmathtruncatemacro{\num}{\col + 4 * \row}
          \node[draw=black, thin, inner sep=0pt] at (\col*4, -\row*4)
            {\includegraphics[width=3.5cm]{img/MUTAG/graph_\num_explanation.pdf}};
        }
      }
    \end{tikzpicture}
    \caption{Grid of MUTAG explanations (top-10 edges)}
    \label{fig:grid-MUTAG-explanations}
\end{figure}

\clearpage
\section{NeuroSAT explainer sweeps}

sample bias $=0.0$
training size = num batches -2
eval size = one batch
loss = paper loss
No L2 reg

Since over all sweeps the mean local ROC-AUC was no higher than $0.502$ and no lower than $0.44$, we select the hyperparameters that maximize the metric.

We conclude that regardless of exact hyperparameters, the NeuroSAT model does not succinctly learn UNSAT cores.


TODO: After these hyperparameters had been established, we ran an additional test for 50 and 70 epochs, but only achieved a consistent ROC-AUC of 0.5.


TODO: THIS ONLY CONSIDERS AUROC SCORE!!! EDGES ARE 0 FOR ALL PREDICTIONS AFTER A CERTAIN TIME!!! Try batchnorm?? Else lower lr, below 0.003...


\begin{table}[h]
    \centering
    \scriptsize
    \begin{tabularx}{\linewidth}{|c|c|c|Y|Y|c|c|c|c|Y|}
    \hline
    \multicolumn{10}{|c|}{\textbf{Hard constraint explainer – Grid search configuration values}} \\ \hline
    entropy reg & epochs & lr mlp & sampled graphs & concat embs & seed & size reg & tT & t0 & complex arch. \\ \hline
    0.1 & 20 & 0.0003 & 1  & False & 75 & 0.1 & 1.0 & 5.0 & False \\
    \textbf{1.0} & \textbf{30} & \textbf{0.003}  & \textbf{5}  & \textbf{True}  & 76 & \textbf{1.0} & \textbf{5.0} &      & \textbf{True}  \\
        &     & 0.01   &  &       &     &      &      &      &       \\ \hline
    \end{tabularx}
    \caption[NeuroSAT hard constraint Sweep]{New configurations from the grid search space not present in the original table. Highlighted values are the best performing. TODO: LR 0.01 TOO LOW - ALL EDGES 0!!}
\end{table}

\begin{table}[h]
    \centering
    \scriptsize
    \begin{tabularx}{\linewidth}{|Y|c|c|Y|Y|c|c|c|c|Y|Y|}
    \hline
    \multicolumn{11}{|c|}{\textbf{Soft constraint explainer}} \\ \hline
    entropy reg & epochs & lr mlp & sampled graphs & concat embs & seed & size reg & tT & t0 & complex arch. & connect. reg\\ \hline
    0.1 & \textbf{20} & 0.0003 & 1 & True & 75 & 0.01 & \textbf{1.0} & 5.0 & True & 0.0 \\ 
    \textbf{1.0} & 30 & 0.003 & 5 & False & 76 & \textbf{0.005} & 5.0 &  & False & \\ 
     &  & \textbf{0.01} &  &  &  &  &  &  &  & \\ \hline
    \end{tabularx}
    \caption[NeuroSAT soft constraint Sweep]{Grid search results over hyperparameter space for the NeuroSAT explainer that uses a soft constraint. Highlighted values are the best performing.}
\end{table}


\section{REMOVE}
\begin{table}[ht]
  \centering
  \scriptsize
  \begin{tabularx}{\textwidth}{l*{4}{X}}   % Now only 4 datasets
  \toprule
  \textbf{} & \multicolumn{4}{c}{\textbf{Explanation AUC}} \\
  \cmidrule{2-5}
  \textbf{Method} & BA-Shapes & BA-Community & Tree-Cycles & Tree-Grid \\
  \midrule
  Our work (inductive) & 0.993$\pm$0.002 & 0.829$\pm$0.008 & 0.109$\pm$0.108 & 0.689$\pm$0.027 \\
  \addlinespace
  \midrule
  \midrule
  \textit{Inference Time (ms)} & 39.0$\pm$1.9 & 25.6$\pm$1.5 & 3.1$\pm$0.3 & 3.4$\pm$0.1 \\
  \bottomrule
  \end{tabularx}
  \caption[REMOVE! PGExplainer with $a=0.08$!]{PGExplainer performance WITH $a=0.08$! (4.8, 64, 4.8, 23) (USED IN SWEEP!)}
  \label{tab:pgexplainer_auc}
\end{table}